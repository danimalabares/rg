\input{/Users/daniel/github/config/preamble-por.sty}%available at github.com/danimalabares/config
%\input{/Users/daniel/github/config/thms-por.sty}%available at github.com/danimalabares/config

\newcommand{\rightlooparrow}{\mathbin{
    \vbox{\openup-10.25pt\halign{\hss$##$\hss\cr\circ\cr\longrightarrow\cr}}
}}

\begin{document}
\bibliographystyle{alpha}

\begin{minipage}{\textwidth}
	\begin{minipage}{1\textwidth}
		\hfill Daniel González Casanova Azuela
		
		{\small Prof. Luis Florit\hfill\href{https://github.com/danimalabares/rg}{github.com/danimalabares/rg}}
	\end{minipage}
\end{minipage}\vspace{.2cm}\hrule

\vspace{10pt}
{\huge Geometria Riemanniana}
\tableofcontents
\section{Aula 1}

\subsection{Lembrando}

\begin{defn}\leavevmode
\textit{\textbf{Variedade diferenciável}}
\begin{enumerate}
\item \(M\) espaço topológico Hausdorff (\(T^2\)), base enumerável. Essas duas condições são equivalentes à existência de partições da unidade.

\item \(M\) localmente euclídeo, i.e. \(\mathcal{A}=\{(\chi_\lambda,U_\lambda)\}\), \(\chi_\lambda:U_\lambda \subset M \to\chi_\lambda(U_\lambda)\subset \mathbb{R}^n\), com \(M=\bigcup_{\lambda}U_\lambda\). Dizemos que \(n\) é a \textit{\textbf{dimensão}} de \(M\).

\item Restringindo dois abertos \(U_\lambda\), \(U_\mu\) com \(U_\lambda \cap U_\mu \neq  \varnothing\), a \textit{\textbf{mudança de coordenadas}} \(\chi_\mu \circ \chi_\lambda^{-1}:\chi_\lambda(U_\lambda \cap U_\mu) \to \chi_\mu(U_\lambda \cap U_\mu)\) deve ser diferenciável. (Nesse curso diferenciável é \(C^\infty\) a menos que especifiquemos).

\item Maximalidade, i.e. \(\mathcal{A}\) é maximal.
\end{enumerate}
\end{defn}

\begin{defn}[Mapa diferenciável]\leavevmode
\(f:M^n \to N^m\) se para todo ponto com cartas \((x,U)\) de  \(M\) e \((y,V)\) de \(N\) o mapa \(y \circ f \circ x^{-1}\) é diferenciável. Denotaremos o conjunto de funções diferenciaveis por \(\mathcal{F}(M,N)\). Em particular \(\mathcal{F}(M):=\mathcal{F}(M,\mathbb{R})\).

\end{defn}

\begin{defn}[Espaço tangente]\leavevmode
\(\mathcal{F}_p(M)\) é o espaço de funções definidas num aberto de \(p\) identificando duas delas se coincidem em qualquer aberto contendo \(p\).

\[T_pM:=\{v \in \mathcal{F}_p(M)^*:v(fg)=f(p)v(g)+g(p)v(f)\}\]
\end{defn}

\begin{question}\leavevmode
\(\mathcal{F}_p(M)\) es el stalk de la gavilla de funciones suaves? Qué pasa si definimos algo como las derivaciones en \(\mathcal{F}(U)\).

A la hora de definir base de \(T_pM\) con los operadores \(\partial_i\) necesitamos fijar una carta, así que en realidad no hay una base canónica de \(T_pM\).
\end{question}

\begin{defn}[Diferencial de uma função]\leavevmode
\[df_p:T_pM \to T_{f(p)}N\]
definida para \(g \in T_{f(p)}N\) como
\[df_p(v)(g)=v(g \circ f)\]
\end{defn}

\begin{remark}\leavevmode
A regra da cadeia é uma tautologia dessa definição!
\end{remark}

\begin{defn}[Base canônica do espaço tangente]\leavevmode
Definimos
\[\partial_i |_{p}=\frac{\partial }{\partial x_i}\Big|_{p}\in T_pM\]
como, para \(g \in T_pM\),
\[\frac{\partial }{\partial x_i}\Big|_{p}(g)=\frac{\partial (g \circ x^{-1})}{\partial u_i}\]
\end{defn}
\begin{exercise}\leavevmode
Mostre que \(\{\partial_1|_{p},\ldots, \partial_n|_{p}\}\) é uma base de \(T_pM\).
\end{exercise}

\begin{proof}[Solution]\leavevmode
Primeiro note que \(\{\partial_i|_{p}\}\) é linearmente independente. Suponha que 
\[\sum a_i \partial_i|_{p} =0\]
Then for every function this gives zero, so in particular for coordinate functions \(x_i:U \to \mathbb{R}\), so
\[0=\Big(\sum a_i \partial_i\Big)x_j=\sum a_i \delta_{ij}=a_j\qquad \text{for all \(j\).} \]
Now let's check \(\operatorname{span}\partial_i|_{p}=T_pM\). Choose a vector \(v \in T_pM\) and let
\[w:=v-\sum_{i}v(x_i)\partial_i|_{p}.\]
We wish to show that \(w=0\).

Then there's the following trick: a function \(g:\mathbb{R} \to \mathbb{R}\) with \(g(0)=0\) can be written \(g(t)=th(t)\) for some continuous function \(h\) (subexercise: construct \(h\), it's an integral). So if we define \(\tilde{g}(t)=g(t)-g(0)\) we can write for any \(g: \mathbb{R} \to \mathbb{R}\) (without asking that \(g(0)=0\)) just \(g(t)=g(0)+th(t)\)

\begin{thing6}{Subexercise}\leavevmode
Mostre que para toda \(g:\mathbb{R} \to \mathbb{R}\) existe \(h: \mathbb{R} \to \mathbb{R}\) contínua tal que \(g(t)=g(0)-th(t)\). \textbf{Solution.} Let \(m_x:\mathbb{R} \to \mathbb{R}\) be the function that multiplies \(t\) times a fixed number \(x\). Notice that, for a fixed \(x\), by fundamental theorem of Calculus
\[\int_0^1\frac{d}{dt}(g \circ m_x)(t)dt=g(x)-g(0)\]
and also
\[\int_0^1\frac{d}{dt}(g \circ m_x)(t)dt=\int_0^1 g'(xt)\cdot x=x \int_0^1 g'(xt)dt\]
Then we define
\[h(x):=\int_0^1g'(xt)dt\]
and immediately we get \(g(x)=g(0)-xh(x)\).
\end{thing6}
\begin{thing6}{Subsubexercise}\leavevmode
	Now do that for \(g:\mathbb{R}^n\to \mathbb{R}\). I think the correct claim is that there exists \(h:\mathbb{R}^n\to \mathbb{R}^n\) such that for every \(\vec{x} \in \mathbb{R}^n\) we have \(g(\vec{x})=g(\vec{0})+\vec{x}\cdot h(\vec{x})\). \textbf{Solution.} Now \(m_x\) multiplies the vector \(x\) times the real number \(t\), it is a function \(m_x: \mathbb{R}\to \mathbb{R}^n\). We get
\[\int_0^1 \frac{d}{dt}(g \circ m_x)(t)dt=g(\vec{x})-g(\vec{0}).\]
And also
\[\int_0^1 \frac{d}{dt}(g \circ m_x)(t)dt=\int_0^1 \nabla_{t \vec{x}} g\cdot \vec{x} dt=\int_0^1 \sum \frac{\partial }{\partial x_i}g\Big|_{t\vec{x}}x_i dt=\sum x_i \cdot  \int_0^1 \frac{\partial g}{\partial x_i}\Big|_{t \vec{x}}.\]
Definimos
\[h(\vec{x}):=\left(\int_0^1 \frac{\partial g}{\partial x_1}\Big|_{t \vec{x}}dt, \ldots , \int_0^1 \frac{\partial g}{\partial x_n}\Big|_{t \vec{x}}dt\right) \]
\end{thing6}

{\color{6}\bfseries Back to the original exercise…}\hspace{.5em}Let's try to use this trick to conclude that \(w(g)=0\) for all \(g \in \mathcal{F}_p\). Since it's a local statement I just suppose that \(g\) is a function  \(g: \mathbb{R}^n \to \mathbb{R}\). Then there is a function \(h:\mathbb{R}^n \to \mathbb{R}^n\) such that for every \(x \in \mathbb{R}^n\), \(g(x)=g(0)+x\cdot h(x)\).

Right so remember that I chose an arbitrary vector \(v \in T_pM\) and defined \(w=v-\sum v(x_i)\partial_i|_{p}\). I can see that \(w(x_i)=0\) for all coordinate functions \(x_i\). But also for \(g\) as above I get
\begin{align*}
w(g)&=w(g(0)+x\cdot h(x))=w(x \cdot  h(x))=w\left(\sum x_i h_i(x)\right) =\sum w(x_i h_i(x))\\
&=\sum \cancelto{0}{w(x_i)}h_i(x)+x_ih_i(x)
\end{align*}
and the second term also vanishes if we suppose that the coordinates of our point, \(x_i\), are all zero. {\color{2}Which makes me think: I think that's the point of the trick, that it somehow manages to put the coordinates of the point inside the whole thing, and then we can suppose the coordinates are 0 and simplify everything}.
\end{proof}

\begin{defn}[Fibrado tangente]\leavevmode
Como os \(\mathcal{F}_p(M)\) são disjuntos, porque \(M\) é Hausdorff, os espaços tangentes são disjuntos para pontos distintos.
\[TM:=\bigsqcup_{p \in M}T_pM\]
com a estrutura diferenciável que você já conhece.

A projeção natural \(\pi:TM \to M\) é uma sumersão no sentido da seguinte definição. (Exercício?)
\end{defn}

\begin{defn}[Imersão e sumersão]\leavevmode
\begin{enumerate}
\item Imersão se para todo \(p \in M\), \(df_p\) é injetiva (e isso implica que \(n \leq  m\)).
\item \textit{\textbf{Sumersão}} de \(df_p\) é sobrejetiva para todo \(p\), implicaq ue \( n \geq  m\).
\item \textit{\textbf{Difeomorfismo local}} se para todo ponto \(df_p\) é um isomorfismo. Isso é equivalente a que para todo ponto existe um aberto tal que \(f|_{U}:U \to V\) é um difemorfismo (teo. função inversa). (Checar.)
\end{enumerate}
\end{defn}

Note que \(f: M \to N\) contínua é como dizer que a topologia induzida por \(f\), \(\tau_f \subset \tau_M\). Mas a igualdade nem sempre tem (e.g. figura 8). \(f\) é um \textit{\textbf{mergulho }} se \(\tau_f = \tau_M\). Isso é equivalente a que \(f(M) \subset N\) seja uma subvariedade e \(f:M \overset{\operatorname{difeo}}{\simeq}f(M)\subset N\).

\begin{defn}[Campo coordenado]\leavevmode
Numa vizinhança \(U\) de \(p\),
\begin{align*}
	\partial : U &\longrightarrow TU\subset TM \\
	p &\longmapsto \frac{\partial }{\partial x_i}\Big|_{p}\in T_pM
\end{align*}
\end{defn}

\begin{remark}\leavevmode
Podemos quase extender esse campo. Num aberto \(V \subset U\) cujo fecho \(\bar{V} \subset U\). Pega a coberta \(\{ M \setminus \bar{V}, U\}\). Então existe part. unidade  \((\xi,\varphi)\). Por definição, \(\varphi|_{V} =1\). Defina \(x= \varphi \partial_i\).
\end{remark}

\begin{defn}[Fibrado vetorial]\leavevmode
Um \textit{\textbf{fibrado vetorial }} \(E^k\) sobre \(M^n\) de posto \(k \in \mathbb{N} \cup  \{0\}\) é
\begin{enumerate}
\item \(\pi:E \to M^n\) submersão sobrejetiva.
\item \(\forall  p \in M\), \(E_p = \pi^{-1}(p)\) é um \(\mathbb{R}\)-e.v. de dimensão \(k\).
\item \(\forall p \in M\), existe \(p \in U \subset M\) y \(\varphi_U\) tal que
	\begin{enumerate}
\item  \(\varphi_U: \pi^{-1}(U) \overset{\operatorname{dif}}{\simeq}U \times \mathbb{R}^k\).
\item \(\varphi_U\) conmuta con la proyección, i.e.
	\[\begin{tikzcd}
	\pi^{-1}(U)\arrow[r,"\varphi_U"]\arrow[d,"\pi",swap]&U \times \mathbb{R}^k\arrow[dl,"\pi_1"]\\
	U
	\end{tikzcd}\]
	\item \(\forall  q \in U\), \(\varphi|_{E_q}:E_q \to \{q\} \times \mathbb{R}^k\cong\mathbb{R}^k\) é um isomorfismo linear.
		\end{enumerate}

	Isso é equivalente a pedir que exista um \textit{\textbf{atlas trivializante}} de \(E\). É \(\{(\varphi,\underbrace{\pi(U)}_{\subseteq E}:U \in \Lambda \subset \tau_M\}\) es decir una familia de abiertos en \(E\) indexada por una familia de abiertos de \(M\). Considere dos de estos abiertos con \(W:=U \cap V \neq  \varnothing\).
	\[\begin{tikzcd}
		\varphi_U|_{\pi(W)}\pi^{-1}(W)\arrow[r]& W \times \mathbb{R}^k\arrow[d]\\
		\varphi_V|_{\pi^{-1}(W)}\pi^{-1}(W)\arrow[r]&W \subset \mathbb{R}^k
	\end{tikzcd}\]
onde estamos parametrizando numa variedade! Ou seja, implícitamente estamos pegando cartas nela, mas podemos deixá-lo assim.

Temos as funções de transição
\[\varphi_{VU}=\varphi_V \circ \varphi_U^{-1}|_{W\times\mathbb{R}^k}:W\times\mathbb{R}^k \to W \times \mathbb{R}^k\]
que realmente estão determinadas por a parte linear:
\[\varphi_{VU}(Q,v)=(Q,\xi_{VU}(Q)(v)\]
onde
\[\xi_{VU}:W \to \mathsf{GL}(k,\mathbb{R})\]
e são chamadas de \textit{\textbf{funções de transição}} de \(E\). Elas satisfacem
\[\xi_{VU}\circ\xi_{SV}=\xi_{SU}\qquad \text{cocycle condition} \]
\[\text{no seria…} \qquad \xi_{VU}\circ\xi_{US}=\xi_{VS}\]
Então podemos formar um fibrado vetorial a partir das funções de transição só.
\end{enumerate}
\end{defn}

\section{Exercícios de do Carmo}

\subsection{Capítulo 0}
\begin{thing4}{Exercise 2}\label{exer:2}\leavevmode
Prove que o fibrado tangente de uma variedade diferenciável \(M\) é orientável (mesmo que \(M\) não seja).
\end{thing4}

\begin{proof}[Solution]\leavevmode
Es porque la diferencial de los cambios de coordenadas está dada por la identidad y una matriz lineal. Sí, porque por definición las trivializaciones locales de \(TM\) preservan la primera coordenada \textbf{y} son isomorfismos lineales en la parte del espacio vectorial. Entonces queda que 
\[d(\varphi_U \circ \varphi_V^{-1})=\left(\begin{array}{@{}c|c@{}}
\operatorname{Id}&0\\
\hline
0&\xi \in \mathsf{GL}(n)\end{array}\right)\]
pero no estoy seguro de por qué \(\xi\) preservaría orientación, i.e. que tenga determinante positivo… a menos de que… 
\end{proof}

\subsection{Capítulo 1}

\begin{thing4}{Exercise 1}\label{exer:1}\leavevmode
Prove que a aplicação antípoda \(A:S^n \to S^n\) dada por \(A(p)=-p\) é uma isometria de \(S^n\). Use este fato para introduzir uma métrica Riemanniana no espaço projetivo real \(\mathbb{R}P^{n}\) tal que a projeção natural \(\pi: S^n \to \mathbb{R}P^{n}\) seja uma isometria local.
\end{thing4}
\begin{proof}[Solution]\leavevmode
	Lembre que a métrica de \(S^n\) é a induzida pela métrica euclidiana, onde pensamos que \(T_pS^n \hookrightarrow T_p\mathbb{R}^{n+1}\). É claro que \(A\) é uma isometría de \(\mathbb{R}^n\), pois ela é a sua derivada (pois ela é linear), de forma que \(\left<v,w\right>_p=\left<-v,-w\right>_{A(p)}=\left<v,w\right>_{-p}\).

	É um fato geral que se as transformações de coberta preservam a métrica, obtemos uma métrica no quociente de maneira natural, i.e. para dois vetores \(v,w\in T_p\mathbb{R}P^n\) definimos \(\left<v,w\right>_p^{\mathbb{R}P^n}:=\left<\tilde{v},\tilde{w}\right>_{\tilde{p} \in \pi^{-1}(p)}\).

Para ver que a projeção natural é uma isometria local basta ver que a diferencial de \(A\) é um isomorfismo em cada ponto. Mas como ela é \(-A\), isso é claro.
\end{proof}

\section{Aula 2}

\begin{defn}\leavevmode
Um \textit{\textbf{fibrado vetorial}} é uma submersão sobrejetora
\[\pi:E \to M\]
onde \(\pi\) é a \textit{\textbf{projeção}},  \(E\) o \textit{\textbf{espaço total}} e \(M\) a \textit{\textbf{base}}. Satisfazendo
\begin{enumerate}
\item \(E\) possui um \textit{\textbf{atlas trivializante}}, i.e. para todo \(p \in M\) existe \(U \ni p\) aberto e carta
		\[\varphi: \pi^{-1}(U) \overset{\operatorname{difeo}}{\to} U \times \mathbb{R}^k\]
		tal que
	\begin{itemize}
	\item \(\pi \circ \varphi_U = \pi|_{\pi^{-1}(U)}\) 
	\item Se \(W=U \cap V \neq  \varnothing\), 
		\begin{align*}
			\varphi_V \circ \varphi_U^{-1} |_{W \times \mathbb{R}^k}: W \times \mathbb{R}^k &\longrightarrow W \times \mathbb{R}^k \\
			(p,v) &\longmapsto (p,\xi_W(p)(v))
		\end{align*}
	onde pedimos que \(\xi_{VU}:W \to \mathsf{GL}(k,\mathbb{R})\), e chamamos esas funções de \textit{\textbf{funções de transição}} de \(E\).
	\end{itemize}
\end{enumerate}
\end{defn}
Note que as fibras são espaços vetoriais: para \(Q \in U\), \(E_Q\overset{\operatorname{def}}{=}\pi^{-1}(Q) \subset E\). Pegue dois elementos \(x, y \in E_Q\). Definimos a soma deles a traves de
\[\varphi(x+y)=(Q,\bar{\varphi} (x)+\bar{\varphi} (y))=(Q,\bar{\varphi} (x+y))\]
onde \(\bar{\varphi} \) é a parte ``linear". Note que isso faz automaticamente que as trivializações sejam lineares nas fibras, i.e. \(E_p \to \{p\}\times \mathbb{R}^k\) linear.

\begin{defn}\leavevmode
As \textit{\textbf{seções de \(E\)}} são
\[\Gamma(E)= \left\{ \begin{tikzcd}\lambda:M\arrow[r]\arrow[rd,"\operatorname{id}"]&  E \arrow[d]\\& M\end{tikzcd} \right\} \]
\end{defn}
\begin{question}\leavevmode
Existe uma coleção de \(k\) seções que são uma base de \(T_pM\) em cada ponto? Não.
\end{question}
\begin{remark}\leavevmode
Existe uma base de seções iff \(E \cong M \times \mathbb{R}^k\). Mas isso ainda nem tem sentido…
\end{remark}

\begin{defn}\leavevmode
Um \textit{\textbf{mapa de fibrados}} é
\[\begin{tikzcd}
	F: E\arrow[r]\arrow[d,"\pi"]&  E \arrow[d,"\pi'"]\\
	M \arrow[r,"f"]& M'
\end{tikzcd}\]
que é linear nas fibras, i.e.
 \[F|_{E_Q}: E_Q \to E'_{f(Q)}.\]
 \(F\) é um \textit{\textbf{isomorfismo}} de fibrados vetoriais iff \(F\) é um difeomorfismo e um mapa de fibrados. (Obviamente isso implica que a inversa é um mapa de fibrados.)
\end{defn}
\begin{remark}\leavevmode
Todo fibrado vetorial possui uma base \textit{local} de seções. Porque pego uma base em \(U \times \mathbb{R}^k\) numa trivialização local e pusho ela pra \(\pi^{-1}(U)\).
\end{remark}

\begin{example}[Fibrado dual]\leavevmode
A observação anterior nos da um jeito super simples de construir o fibrado dual: para cada trivializacão local, e para cada ponto definimos a base dual do espaço vetorial original no ponto, e é isso, tudo segue.

Outros exemplos podem ser construidos do mesmo jeito: \(\operatorname{End}(E)\), \(\Lambda^{r}(\mathbb{V})\). A ideia e que ``a álgebra linear pode ser fibralizada por causa de que temos bases locais".
\end{example}

\begin{example}\leavevmode

Outro exemplo, embora não é um fibrado vetorial, é o conjunto de orientações de \(\mathbb{V}\), \(\mathcal{O}(\mathbb{V} ):=\{\text{bases de \(\mathbb{V}\)} \}\Big/ \sim\). Definimos um  \textit{\textbf{fibrado orientável}} se \(\mathcal{O}(E)\) tem uma seção global. Isso se traduz a que em cada ponto exista uma carta tal que a orientaçao.. seja compatível?

Também podemos definir \(M\) \textit{\textbf{orientável}} se \(TM\) orientavel \textit{como fibrado}. \(TM\) sempre é orientavel \textit{como variedade}  porque \(T TM\) é orientável \textit{como fibrado}.
\end{example}

\begin{example}[Tensores=aplicações multilineares]\leavevmode
Pega \(\mathbb{V}\) esp. vect e considere os tensores \(\{T: \mathbb{V} \times \ldots \times\mathbb{V} \to \mathbb{R}\}:=\operatorname{Multi}(E)\). As seções disso são \(\mathfrak{X}^r(E)\). No caso do fibrado tangente se denotam \(T \in \mathfrak{X}^r(M)\overset{\operatorname{def}}{=}\Gamma TM\), e se chamam \textit{\textbf{campos tensoriais}}.
\end{example}

\subsubsection{Tensores}

A ver la notación es que \(T \in \mathfrak{X}^r\)

\begin{upshot}[del ejercicio]\leavevmode
Que es lo mismo pensar en un operador que come campos vectoriales y da funciones, o un campo, una cosa que en cada punto me da un operador que come vectores.
\end{upshot}

\begin{thing6}{Siguiente cosa}[A dupla personalidade dos campos vetoriais]\leavevmode
Que podemos pensar que los campos vectoriales son derivaciones. \(\hat{X}:\mathcal{F}(M) \to \mathcal{F}(M)\). Sí porque un campo de vectores en un punto puede ser evaluado en una función y da un número, y bueno satisface Leibniz.
\end{thing6}

Va otra construcción:

\(E\), pega \(\Lambda^{r}(E)\), os mapas \(r\)-alternantes de \(E\), que é um fibrado vetorial. As seções dele,  \(\Gamma(\Lambda^{r}E)\). No caso do fibrado tangente, \(\Omega^r(M):= \Lambda^{r}(TM)\). Entonces a ver de nuevo: pega \(\omega \in \Lambda^{r}TM\). En cada punto me da una aplicación \(e\)-multiniear alternante, pero también lo puedo ver como un mapa \(\omega: \mathfrak{X} M\times .. \times \mathfrak{X}M \to \mathcal{F}M\).

\begin{exercise}\leavevmode
\(M^n\) é orientável \(\iff\) \(\Lambda^{n}M\) possui seção nunca nula.
\end{exercise}

Lembre que \(\Omega_c^n(M)\) é o espaço de formas cujo suporte tem fecho compacto.

\begin{remark}\leavevmode
\(M\) orientada \(\implies\) integral está bem definida. Sim, porque o teorema de mudança de variáveis diz que para \(\varphi: U \to V\), \(\omega \in \Omega^{n}(V)\), \(\int_U \varphi^*\omega=\text{*sinal!*}\int_V \omega \). Então para que não se faça uma bagunça precisamos que os determinantes das mudanças de coordenadas coincidam.
\end{remark}

\begin{defn}[Fibrado pullback]\leavevmode
\[\begin{tikzcd}
	f^* (E)\arrow[r,"\pi_2"]\arrow[d,"\pi_1",swap]&E\arrow[d,"\pi"]\\
	M\arrow[r,swap,"f"]&N
\end{tikzcd}\]
onde
\[f^*(E)=\{(p,v) \in M \times E: \pi(v)=f(p)\}\]
(Note que botamos o \(p\) em \((p,v)\) para obter que o espaço total de \(f^*(E)\) seja uma coleção \textit{disjunta} de fibras.)
\end{defn}
Essa é uma definição ótima. Note que \(\pi_2\) é um mapa de fibrados que aparece de graça. (Não é um isomorfismo.)

\begin{remark}\leavevmode
O pullback é mágico porque ele leva todas as propriedades de \(E\) como curvatura, conexão, etc.
\end{remark}

\begin{remark}\leavevmode
Se \(f\) é constante obtemos o fibrado trivial.
\end{remark}

\begin{remark}\leavevmode
Pega  \(\xi \Gamma(f^* E)\). Então temos para \(p \in M\) um elemento \(\xi(p)=(p, \bar{\xi}(p)) \). Então olha
\[\begin{tikzcd}
	\bar{\xi} :M \arrow[r]\arrow[dr,"f"]&  R \arrow[d,"\pi"]\\& N
\end{tikzcd}\]
então essas seçnes se chaman de \(\mathfrak{X}_f \cong \Gamma(f^*(E))\) \textit{\textbf{seções ao longo de \(f\)}}.
\end{remark}
Entonces el punto es que, por construcción cada sección del pullback me da un elemento en el otro vb y de ahi quiero que la proyección me devuelva \(f\).

\begin{defn}\leavevmode
Dos campos \(X \in \mathfrak{X}(M)\) e \(Y \in \mathfrak{X}(N)\) están \textit{\textbf{\(f\)-relacionados}} \(X \overset{f}{\sim}Y\) se \(Y \circ f =f_*X\) donde \(f:M \to N\). Pero pérame porque a mí me habían dicho que no siempre \(f_*X\) está bien definido. Ah, porque aquí \(f_*X\) es un campo \textit{ao longo de \(f\)}; así \textit{siempre} está bien definido. Entonces tiene sentido la definición y el ejercicio:
\end{defn}

\begin{exercise}\leavevmode
	\(X_1 \overset{f}{\sim}X_2\) \(\implies \) \([X_1,X_2] \overset{f}{\sim}[Y_1,Y_2]\)
\textbf{Hint.} Pensa que um campo é uma coisa que pega uma função e me da uma função.
\end{exercise}

\subsection{Grupos de Lie}

\begin{defn}\leavevmode
Um \textit{\textbf{grupo de Lie}} é um grupo \(G\) que é uma variedade diferenciável tal que
\[\cdot :G \times G \to G \qquad \qquad \leavevmode^{-1}:G \to G\]
são diferenciaveis.
\end{defn}

Os grupos de Lie tem um monte de difeomorfismos dados pela multiplicacão a esquerda: \(h \in G \rightsquigarrow L_h:G \to G\), \(L_h(g)=h\cdot g\). Como \(L_{h^{-1}}\circ L_h=\operatorname{Id}\), \(L_h \in \operatorname{Dif}G\).

\begin{exercise}\leavevmode
\(v \in T_eG\), \(X_v(g)=d(L_g)_e(v) \in T_gG\), \(\implies\) \(X_v \in \mathfrak{X}(G)\). \textbf{Note} que vai precisar usar que o produto do grupo é diferenciável.
\end{exercise}
E aí fica que uma base \(\{v_i\}\subset T_eG\) nos da uma base global de seções. Em outras palavras, o fibrado tangente de um grupo de Lie é trivial. Isso é rarísimo, uma variedade com fibrado tangente trivial, se chama variedade paralelizável.

\begin{remark}\leavevmode
\(\forall g \in G, X_v \overset{L_g}{\sim}X_v\) para todo \(v \in T_eG\). Conta:
\[\text{tu pode} \]
Mas ainda, se ele está \(L_g\) relacionado com ele mesmo para todo \(g \in G\) (isso se chama ser \textit{\textbf{invariante a esqueda}}), então ele é um \(X_v\) para algum \(v\). Conta:
\[\text{pode} \]
Então ai fica essa equivalência, e ademais, se pegamos \(v,w \in T_eG\) podemos pensar em \(X_v,X_w\), e \textit{definimos} \(X_{[v,w]}:=[X_v,X_w]\). E ai obtemos a \textit{\textbf{álgebra de Lie}} de \(G\), que é \((T_eG,[,]):=\mathfrak{g}\).
\end{remark}

\begin{thing6}{Mais um}\leavevmode
Pegue \(X \in \mathfrak{g}\) e \(\gamma\) curva integral de \(X\) pasando por \(e\), diagmos \(\gamma(0)=e\). Prove que
\begin{enumerate}
\item Se \(\varphi_t\) é o fluxo de \(X\) \(\implies\) \(L_g \circ\varphi_t = \varphi_t\circ L_g\), \(\varphi_t = R_{\gamma(t)}\).
\item \(\gamma\) é homomorfismo de grupos \(\mathbb{R} \to G\). Isso permete definir \(\operatorname{exp}^G: \mathfrak{g} \to G\) dada por \(\operatorname{exp}^G(X)=\gamma(1)\). Prove que \(\operatorname{exp}^G(tX)=\gamma(t)\).

\textbf{Hint.} O último implica os outros.
\end{enumerate}
\end{thing6}

\end{document}
