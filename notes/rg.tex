\input{/Users/daniel/github/config/preamble-por.sty}%available at github.com/danimalabares/config
%\input{/Users/daniel/github/config/thms-por.sty}%available at github.com/danimalabares/config

\newcommand{\rightlooparrow}{\mathbin{
    \vbox{\openup-10.25pt\halign{\hss$##$\hss\cr\circ\cr\longrightarrow\cr}}
}}

\begin{document}
\bibliographystyle{alpha}

\begin{minipage}{\textwidth}
	\begin{minipage}{1\textwidth}
		\hfill Daniel González Casanova Azuela
		
		{\small Prof. Luis Florit\hfill\href{https://github.com/danimalabares/rg}{github.com/danimalabares/rg}}
	\end{minipage}
\end{minipage}\vspace{.2cm}\hrule

\vspace{10pt}
{\huge Geometria Riemanniana}
\tableofcontents
\section{Aula 1}

\subsection{Lembrando}

\begin{defn}\leavevmode
\textit{\textbf{Variedade diferenciável}}
\begin{enumerate}
\item \(M\) espaço topológico Hausdorff (\(T^2\)), base enumerável. Essas duas condições são equivalentes à existência de partições da unidade.

\item \(M\) localmente euclídeo, i.e. \(\mathcal{A}=\{(\chi_\lambda,U_\lambda)\}\), \(\chi_\lambda:U_\lambda \subset M \to\chi_\lambda(U_\lambda)\subset \mathbb{R}^n\), com \(M=\bigcup_{\lambda}U_\lambda\). Dizemos que \(n\) é a \textit{\textbf{dimensão}} de \(M\).

\item Restringindo dois abertos \(U_\lambda\), \(U_\mu\) com \(U_\lambda \cap U_\mu \neq  \varnothing\), a \textit{\textbf{mudança de coordenadas}} \(\chi_\mu \circ \chi_\lambda^{-1}:\chi_\lambda(U_\lambda \cap U_\mu) \to \chi_\mu(U_\lambda \cap U_\mu)\) deve ser diferenciável. (Nesse curso diferenciável é \(C^\infty\) a menos que especifiquemos).

\item Maximalidade, i.e. \(\mathcal{A}\) é maximal.
\end{enumerate}
\end{defn}

\begin{defn}[Mapa diferenciável]\leavevmode
\(f:M^n \to N^m\) se para todo ponto com cartas \((x,U)\) de  \(M\) e \((y,V)\) de \(N\) o mapa \(y \circ f \circ x^{-1}\) é diferenciável. Denotaremos o conjunto de funções diferenciaveis por \(\mathcal{F}(M,N)\). Em particular \(\mathcal{F}(M):=\mathcal{F}(M,\mathbb{R})\).

\end{defn}

\begin{defn}[Espaço tangente]\leavevmode
\(\mathcal{F}_p(M)\) é o espaço de funções definidas num aberto de \(p\) identificando duas delas se coincidem em qualquer aberto contendo \(p\).

\[T_pM:=\{v \in \mathcal{F}_p(M)^*:v(fg)=f(p)v(g)+g(p)v(f)\}\]
\end{defn}

\begin{question}\leavevmode
\(\mathcal{F}_p(M)\) es el stalk de la gavilla de funciones suaves? Qué pasa si definimos algo como las derivaciones en \(\mathcal{F}(U)\).

A la hora de definir base de \(T_pM\) con los operadores \(\partial_i\) necesitamos fijar una carta, así que en realidad no hay una base canónica de \(T_pM\).
\end{question}

\begin{defn}[Diferencial de uma função]\leavevmode
\[df_p:T_pM \to T_{f(p)}N\]
definida para \(g \in T_{f(p)}N\) como
\[df_p(v)(g)=v(g \circ f)\]
\end{defn}

\begin{remark}\leavevmode
A regra da cadeia é uma tautologia dessa definição!
\end{remark}

\begin{defn}[Base canônica do espaço tangente]\leavevmode
Definimos
\[\partial_i |_{p}=\frac{\partial }{\partial x_i}\Big|_{p}\in T_pM\]
como, para \(g \in T_pM\),
\[\frac{\partial }{\partial x_i}\Big|_{p}(g)=\frac{\partial (g \circ x^{-1})}{\partial u_i}\]
\end{defn}
\begin{exercise}\leavevmode
Mostre que \(\{\partial_1|_{p},\ldots, \partial_n|_{p}\}\) é uma base de \(T_pM\).
\end{exercise}

\begin{proof}[Solution]\leavevmode
Primeiro note que \(\{\partial_i|_{p}\}\) é linearmente independente. Suponha que 
\[\sum a_i \partial_i|_{p} =0\]
Then for every function this gives zero, so in particular for coordinate functions \(x_i:U \to \mathbb{R}\), so
\[0=\Big(\sum a_i \partial_i\Big)x_j=\sum a_i \delta_{ij}=a_j\qquad \text{for all \(j\).} \]
Now let's check \(\operatorname{span}\partial_i|_{p}=T_pM\). Choose a vector \(v \in T_pM\) and let
\[w:=v-\sum_{i}v(x_i)\partial_i|_{p}.\]
We wish to show that \(w=0\).

Then there's the following trick: a function \(g:\mathbb{R} \to \mathbb{R}\) with \(g(0)=0\) can be written \(g(t)=th(t)\) for some continuous function \(h\) (subexercise: construct \(h\), it's an integral). So if we define \(\tilde{g}(t)=g(t)-g(0)\) we can write for any \(g: \mathbb{R} \to \mathbb{R}\) (without asking that \(g(0)=0\)) just \(g(t)=g(0)+th(t)\)

\begin{thing6}{Subexercise}\leavevmode
Mostre que para toda \(g:\mathbb{R} \to \mathbb{R}\) existe \(h: \mathbb{R} \to \mathbb{R}\) contínua tal que \(g(t)=g(0)-th(t)\). \textbf{Solution.} Let \(m_x:\mathbb{R} \to \mathbb{R}\) be the function that multiplies \(t\) times a fixed number \(x\). Notice that, for a fixed \(x\), by fundamental theorem of Calculus
\[\int_0^1\frac{d}{dt}(g \circ m_x)(t)dt=g(x)-g(0)\]
and also
\[\int_0^1\frac{d}{dt}(g \circ m_x)(t)dt=\int_0^1 g'(xt)\cdot x=x \int_0^1 g'(xt)dt\]
Then we define
\[h(x):=\int_0^1g'(xt)dt\]
and immediately we get \(g(x)=g(0)-xh(x)\).
\end{thing6}
\begin{thing6}{Subsubexercise}\leavevmode
	Now do that for \(g:\mathbb{R}^n\to \mathbb{R}\). I think the correct claim is that there exists \(h:\mathbb{R}^n\to \mathbb{R}^n\) such that for every \(\vec{x} \in \mathbb{R}^n\) we have \(g(\vec{x})=g(\vec{0})+\vec{x}\cdot h(\vec{x})\). \textbf{Solution.} Now \(m_x\) multiplies the vector \(x\) times the real number \(t\), it is a function \(m_x: \mathbb{R}\to \mathbb{R}^n\). We get
\[\int_0^1 \frac{d}{dt}(g \circ m_x)(t)dt=g(\vec{x})-g(\vec{0}).\]
And also
\[\int_0^1 \frac{d}{dt}(g \circ m_x)(t)dt=\int_0^1 \nabla_{t \vec{x}} g\cdot \vec{x} dt=\int_0^1 \sum \frac{\partial }{\partial x_i}g\Big|_{t\vec{x}}x_i dt=\sum x_i \cdot  \int_0^1 \frac{\partial g}{\partial x_i}\Big|_{t \vec{x}}.\]
Definimos
\[h(\vec{x}):=\left(\int_0^1 \frac{\partial g}{\partial x_1}\Big|_{t \vec{x}}dt, \ldots , \int_0^1 \frac{\partial g}{\partial x_n}\Big|_{t \vec{x}}dt\right) \]
\end{thing6}

{\color{6}\bfseries Back to the original exercise…}\hspace{.5em}Let's try to use this trick to conclude that \(w(g)=0\) for all \(g \in \mathcal{F}_p\). Since it's a local statement I just suppose that \(g\) is a function  \(g: \mathbb{R}^n \to \mathbb{R}\). Then there is a function \(h:\mathbb{R}^n \to \mathbb{R}^n\) such that for every \(x \in \mathbb{R}^n\), \(g(x)=g(0)+x\cdot h(x)\).

Right so remember that I chose an arbitrary vector \(v \in T_pM\) and defined \(w=v-\sum v(x_i)\partial_i|_{p}\). I can see that \(w(x_i)=0\) for all coordinate functions \(x_i\). But also for \(g\) as above I get
\begin{align*}
w(g)&=w(g(0)+x\cdot h(x))=w(x \cdot  h(x))=w\left(\sum x_i h_i(x)\right) =\sum w(x_i h_i(x))\\
&=\sum \cancelto{0}{w(x_i)}h_i(x)+x_ih_i(x)
\end{align*}
and the second term also vanishes if we suppose that the coordinates of our point, \(x_i\), are all zero. {\color{2}Which makes me think: I think that's the point of the trick, that it somehow manages to put the coordinates of the point inside the whole thing, and then we can suppose the coordinates are 0 and simplify everything}.
\end{proof}

\begin{defn}[Fibrado tangente]\leavevmode
Como os \(\mathcal{F}_p(M)\) são disjuntos, porque \(M\) é Hausdorff, os espaços tangentes são disjuntos para pontos distintos.
\[TM:=\bigsqcup_{p \in M}T_pM\]
com a estrutura diferenciável que você já conhece.

A projeção natural \(\pi:TM \to M\) é uma sumersão no sentido da seguinte definição. (Exercício?)
\end{defn}

\begin{defn}[Imersão e sumersão]\leavevmode
\begin{enumerate}
\item Imersão se para todo \(p \in M\), \(df_p\) é injetiva (e isso implica que \(n \leq  m\)).
\item \textit{\textbf{Sumersão}} de \(df_p\) é sobrejetiva para todo \(p\), implicaq ue \( n \geq  m\).
\item \textit{\textbf{Difeomorfismo local}} se para todo ponto \(df_p\) é um isomorfismo. Isso é equivalente a que para todo ponto existe um aberto tal que \(f|_{U}:U \to V\) é um difemorfismo (teo. função inversa). (Checar.)
\end{enumerate}
\end{defn}

Note que \(f: M \to N\) contínua é como dizer que a topologia induzida por \(f\), \(\tau_f \subset \tau_M\). Mas a igualdade nem sempre tem (e.g. figura 8). \(f\) é um \textit{\textbf{mergulho }} se \(\tau_f = \tau_M\). Isso é equivalente a que \(f(M) \subset N\) seja uma subvariedade e \(f:M \overset{\operatorname{difeo}}{\simeq}f(M)\subset N\).

\begin{defn}[Campo coordenado]\leavevmode
Numa vizinhança \(U\) de \(p\),
\begin{align*}
	\partial : U &\longrightarrow TU\subset TM \\
	p &\longmapsto \frac{\partial }{\partial x_i}\Big|_{p}\in T_pM
\end{align*}
\end{defn}

\begin{remark}\leavevmode
Podemos quase extender esse campo. Num aberto \(V \subset U\) cujo fecho \(\bar{V} \subset U\). Pega a coberta \(\{ M \setminus \bar{V}, U\}\). Então existe part. unidade  \((\xi,\varphi)\). Por definição, \(\varphi|_{V} =1\). Defina \(x= \varphi \partial_i\).
\end{remark}

\begin{defn}[Fibrado vetorial]\leavevmode
Um \textit{\textbf{fibrado vetorial }} \(E^k\) sobre \(M^n\) de posto \(k \in \mathbb{N} \cup  \{0\}\) é
\begin{enumerate}
\item \(\pi:E \to M^n\) submersão sobrejetiva.
\item \(\forall  p \in M\), \(E_p = \pi^{-1}(p)\) é um \(\mathbb{R}\)-e.v. de dimensão \(k\).
\item \(\forall p \in M\), existe \(p \in U \subset M\) y \(\varphi_U\) tal que
	\begin{enumerate}
\item  \(\varphi_U: \pi^{-1}(U) \overset{\operatorname{dif}}{\simeq}U \times \mathbb{R}^k\).
\item \(\varphi_U\) conmuta con la proyección, i.e.
	\[\begin{tikzcd}
	\pi^{-1}(U)\arrow[r,"\varphi_U"]\arrow[d,"\pi",swap]&U \times \mathbb{R}^k\arrow[dl,"\pi_1"]\\
	U
	\end{tikzcd}\]
	\item \(\forall  q \in U\), \(\varphi|_{E_q}:E_q \to \{q\} \times \mathbb{R}^k\cong\mathbb{R}^k\) é um isomorfismo linear.
		\end{enumerate}

	Isso é equivalente a pedir que exista um \textit{\textbf{atlas trivializante}} de \(E\). É \(\{(\varphi,\underbrace{\pi(U)}_{\subseteq E}:U \in \Lambda \subset \tau_M\}\) es decir una familia de abiertos en \(E\) indexada por una familia de abiertos de \(M\). Considere dos de estos abiertos con \(W:=U \cap V \neq  \varnothing\).
	\[\begin{tikzcd}
		\varphi_U|_{\pi(W)}\pi^{-1}(W)\arrow[r]& W \times \mathbb{R}^k\arrow[d]\\
		\varphi_V|_{\pi^{-1}(W)}\pi^{-1}(W)\arrow[r]&W \subset \mathbb{R}^k
	\end{tikzcd}\]
onde estamos parametrizando numa variedade! Ou seja, implícitamente estamos pegando cartas nela, mas podemos deixá-lo assim.

Temos as funções de transição
\[\varphi_{VU}=\varphi_V \circ \varphi_U^{-1}|_{W\times\mathbb{R}^k}:W\times\mathbb{R}^k \to W \times \mathbb{R}^k\]
que realmente estão determinadas por a parte linear:
\[\varphi_{VU}(Q,v)=(Q,\xi_{VU}(Q)(v)\]
onde
\[\xi_{VU}:W \to \mathsf{GL}(k,\mathbb{R})\]
e são chamadas de \textit{\textbf{funções de transição}} de \(E\). Elas satisfacem
\[\xi_{VU}\circ\xi_{SV}=\xi_{SU}\qquad \text{cocycle condition} \]
\[\text{no seria…} \qquad \xi_{VU}\circ\xi_{US}=\xi_{VS}\]
Então podemos formar um fibrado vetorial a partir das funções de transição só.
\end{enumerate}
\end{defn}

\section{Aula 2}

\subsection{Fibrados vetoriais}

\begin{defn}\leavevmode
Um \textit{\textbf{fibrado vetorial}} é uma submersão sobrejetora
\[\pi:E \to M\]
onde \(\pi\) é a \textit{\textbf{projeção}},  \(E\) o \textit{\textbf{espaço total}} e \(M\) a \textit{\textbf{base}}. Satisfazendo
\begin{enumerate}
\item \(E\) possui um \textit{\textbf{atlas trivializante}}, i.e. para todo \(p \in M\) existe \(U \ni p\) aberto e carta
		\[\varphi: \pi^{-1}(U) \overset{\operatorname{difeo}}{\to} U \times \mathbb{R}^k\]
		tal que
	\begin{itemize}
	\item \(\pi \circ \varphi_U = \pi|_{\pi^{-1}(U)}\) 
	\item Se \(W=U \cap V \neq  \varnothing\), 
		\begin{align*}
			\varphi_V \circ \varphi_U^{-1} |_{W \times \mathbb{R}^k}: W \times \mathbb{R}^k &\longrightarrow W \times \mathbb{R}^k \\
			(p,v) &\longmapsto (p,\xi_W(p)(v))
		\end{align*}
	onde pedimos que \(\xi_{VU}:W \to \mathsf{GL}(k,\mathbb{R})\), e chamamos esas funções de \textit{\textbf{funções de transição}} de \(E\).
	\end{itemize}
\end{enumerate}
\end{defn}
Note que as fibras são espaços vetoriais: para \(Q \in U\), \(E_Q\overset{\operatorname{def}}{=}\pi^{-1}(Q) \subset E\). Pegue dois elementos \(x, y \in E_Q\). Definimos a soma deles a traves de
\[\varphi(x+y)=(Q,\bar{\varphi} (x)+\bar{\varphi} (y))=(Q,\bar{\varphi} (x+y))\]
onde \(\bar{\varphi} \) é a parte ``linear". Note que isso faz automaticamente que as trivializações sejam lineares nas fibras, i.e. \(E_p \to \{p\}\times \mathbb{R}^k\) linear.

\begin{defn}\leavevmode
As \textit{\textbf{seções de \(E\)}} são
\[\Gamma(E)= \left\{ \begin{tikzcd}\lambda:M\arrow[r]\arrow[rd,"\operatorname{id}"]&  E \arrow[d]\\& M\end{tikzcd} \right\} \]
\end{defn}
\begin{question}\leavevmode
Existe uma coleção de \(k\) seções que são uma base de \(T_pM\) em cada ponto? Não.
\end{question}
\begin{remark}\leavevmode
Existe uma base de seções iff \(E \cong M \times \mathbb{R}^k\). Mas isso ainda nem tem sentido…
\end{remark}

\begin{defn}\leavevmode
Um \textit{\textbf{mapa de fibrados}} é
\[\begin{tikzcd}
	F: E\arrow[r]\arrow[d,"\pi"]&  E \arrow[d,"\pi'"]\\
	M \arrow[r,"f"]& M'
\end{tikzcd}\]
que é linear nas fibras, i.e.
 \[F|_{E_Q}: E_Q \to E'_{f(Q)}.\]
 \(F\) é um \textit{\textbf{isomorfismo}} de fibrados vetoriais iff \(F\) é um difeomorfismo e um mapa de fibrados. (Obviamente isso implica que a inversa é um mapa de fibrados.)
\end{defn}
\begin{remark}\leavevmode
Todo fibrado vetorial possui uma base \textit{local} de seções. Porque pego uma base em \(U \times \mathbb{R}^k\) numa trivialização local e pusho ela pra \(\pi^{-1}(U)\).
\end{remark}

\begin{example}[Fibrado dual]\leavevmode
A observação anterior nos da um jeito super simples de construir o fibrado dual: para cada trivializacão local, e para cada ponto definimos a base dual do espaço vetorial original no ponto, e é isso, tudo segue.

Outros exemplos podem ser construidos do mesmo jeito: \(\operatorname{End}(E)\), \(\Lambda^{r}(\mathbb{V})\). A ideia e que ``a álgebra linear pode ser fibralizada por causa de que temos bases locais".
\end{example}

\begin{example}\leavevmode

Outro exemplo, embora não é um fibrado vetorial, é o conjunto de orientações de \(\mathbb{V}\), \(\mathcal{O}(\mathbb{V} ):=\{\text{bases de \(\mathbb{V}\)} \}\Big/ \sim\). Definimos um  \textit{\textbf{fibrado orientável}} se \(\mathcal{O}(E)\) tem uma seção global. Isso se traduz a que em cada ponto exista uma carta tal que a orientaçao.. seja compatível?

Também podemos definir \(M\) \textit{\textbf{orientável}} se \(TM\) orientavel \textit{como fibrado}. \(TM\) sempre é orientavel \textit{como variedade}  porque \(T TM\) é orientável \textit{como fibrado}.
\end{example}

\begin{example}[Tensores=aplicações multilineares]\leavevmode
Pega \(\mathbb{V}\) esp. vect e considere os tensores \(\{T: \mathbb{V} \times \ldots \times\mathbb{V} \to \mathbb{R}\}:=\operatorname{Multi}(E)\). As seções disso são \(\mathfrak{X}^r(E)\). No caso do fibrado tangente se denotam \(T \in \mathfrak{X}^r(M)\overset{\operatorname{def}}{=}\Gamma TM\), e se chamam \textit{\textbf{campos tensoriais}}.
\end{example}

\subsubsection{Tensores}

{\color{6}Isso daqui é como eu acho que deveria ser:} acho que em aula definimos \(\mathfrak{X}^r(M)\) como sendo o conjunto de mapas \(r\)-multilineares \(T:M \to \underbrace{T_pM\times\ldots\times T_pM}_{r\text{ vezes} }\), mas na verdade deveria ser \((T^*M)^r:= \bigotimes_r T^*M\). (Devemos pegar produto tensorial para construir um fibrado vetorial certinho.)

\begin{exercise}\leavevmode
Mostre que os seguintes dois \(\mathcal{F}(M)\)-módulos são isomorfos:
\begin{align*}
	\left\{\begin{aligned}T: M &\longrightarrow (T^*M)^r \\
	p &\longmapsto T(p): (T_pM)^r \longrightarrow \mathbb{R}\qquad \text{\(r\)-\(\mathbb{R}\)-multilinear}
\end{aligned}\right\}
\end{align*}
\begin{align*}
\left\{ \begin{aligned}
\hat{T}:\mathfrak{X}^r(M) &\longrightarrow\mathcal{F}(M)\\
X_1,\ldots,X_r &  \longmapsto \hat{T}: M \longrightarrow \mathbb{R}
\end{aligned}\qquad \text{\(r\)-\(\mathcal{F}(M)\)-multilinear} \right\} 
\end{align*}
\end{exercise}

\begin{proof}[Solução]\leavevmode
	Defina o primeiro conjunto como \(A\) e o segundo como \(B\). Pegue \(T \in A\) e defina
\begin{align*}
	\begin{aligned}
		\hat{T}:\mathfrak{X}^r(M) &\longrightarrow\mathcal{F}(M)\\
		V_1,\ldots,V_r &  \longmapsto 
		\begin{aligned}[t]
			\hat{T}: M &\longrightarrow \mathbb{R} \\
			p &\longmapsto T(p)(V_{1,p},\ldots,V_{r,p})
		\end{aligned}
	\end{aligned}
\end{align*}

Ao contrário, pegue \(\hat{T} \in B\) e defina
\begin{align*}
\begin{aligned}
	T: M &\longrightarrow  (T^*M)^r\\
	 p&\longmapsto \begin{aligned}
	 	T(p): (T_pM)^r &\longrightarrow \mathbb{R} \\
		(v_1,\ldots,v_r) &\longmapsto \hat{T}(V_1,\ldots,V_r)
	 \end{aligned}
\end{aligned}
\end{align*}
onde \(V_i\) é uma extensão de \(v_i\) usando partição da unidade.
\end{proof}

\begin{upshot}[del ejercicio]\leavevmode
Que es lo mismo pensar en un operador que come campos vectoriales y da funciones, o un campo {\color{6}*covectorial*}, una cosa que en cada punto me da un operador que come vectores.
\end{upshot}

\begin{thing6}{Siguiente cosa}[A dupla personalidade dos campos vetoriais]\leavevmode
Que podemos pensar que los campos vectoriales son derivaciones. \(\hat{X}:\mathcal{F}(M) \to \mathcal{F}(M)\). Sí porque un campo de vectores en un punto puede ser evaluado en una función y da un número, y bueno satisface Leibniz.
\end{thing6}

Va otra construcción:

\(E\), pega \(\Lambda^{r}(E)\), os mapas \(r\)-alternantes de \(E\), que é um fibrado vetorial. As seções dele,  \(\Gamma(\Lambda^{r}E)\). No caso do fibrado tangente, \(\Omega^r(M):= \Lambda^{r}(TM)\). Entonces a ver de nuevo: pega \(\omega \in \Lambda^{r}TM\). En cada punto me da una aplicación \(e\)-multiniear alternante, pero también lo puedo ver como un mapa \(\omega: \mathfrak{X} M\times .. \times \mathfrak{X}M \to \mathcal{F}M\).

\begin{exercise}\leavevmode
\(M^n\) é orientável \(\iff\) \(\Lambda^{n}M\) possui seção nunca nula.
\end{exercise}

\begin{proof}[Solution]\leavevmode
\((\implies )\) Em cada ponto \(p \in M\) temos uma base orientada \(\{e_i\}\) de \(T_pM\). Essa base me permite expressar qualquer coleção de \(n\) vetores \(v_1,\ldots,v_n \in T_pM\) como uma matriz \((v^i_{j})\). O determinante dessa matriz é uma \(n\)-forma alternante. 

Note que essa função está bem definida na classe
Definindo  \(\omega_p(v_1,\ldots,v_n)=\det v^i_j\) obtemos uma seção não nula de \(\Lambda^{n}M\).

Para argumentar que essa é uma correspondência suave devemos argumentar que o mapa \(\mathcal{O}(M)=\{\text{bases} \}/\sim \longrightarrow \Lambda^{n}TM\) é suave. Para isso deveríamos olhar para a estrutura diferenciável de \(\mathcal{O}(M)\).

\((\impliedby)\) Pegue \(\omega \in \Lambda^{n}TM\), qualquer ponto \(p \in M\) e uma base \(\{v_i\}\subset T_pM\) tal que \(\omega_p(v_i)=1\). Afirmo que \(p \mapsto [\{v_i\}]\in \mathcal{O}(M)\) é uma seção global de \(\mathcal{O}(M)\).
\end{proof}

Lembre que \(\Omega_c^n(M)\) é o espaço de formas cujo suporte tem fecho compacto.

\begin{remark}\leavevmode
\(M\) orientada \(\implies\) integral está bem definida. Sim, porque o teorema de mudança de variáveis diz que para \(\varphi: U \to V\), \(\omega \in \Omega^{n}(V)\), \(\int_U \varphi^*\omega=\text{*sinal!*}\int_V \omega \). Então para que não se faça uma bagunça precisamos que os determinantes das mudanças de coordenadas coincidam.
\end{remark}

\begin{defn}[Fibrado pullback]\leavevmode
\[\begin{tikzcd}
	f^* (E)\arrow[r,"\pi_2"]\arrow[d,"\pi_1",swap]&E\arrow[d,"\pi"]\\
	M\arrow[r,swap,"f"]&N
\end{tikzcd}\]
onde
\[f^*(E)=\{(p,v) \in M \times E: \pi(v)=f(p)\}\]
(Note que botamos o \(p\) em \((p,v)\) para obter que o espaço total de \(f^*(E)\) seja uma coleção \textit{disjunta} de fibras.)
\end{defn}
Essa é uma definição ótima. Note que \(\pi_2\) é um mapa de fibrados que aparece de graça. (Não é um isomorfismo.)

\begin{remark}\leavevmode
O pullback é mágico porque ele leva todas as propriedades de \(E\) como curvatura, conexão, etc.
\end{remark}

\begin{remark}\leavevmode
Se \(f\) é constante obtemos o fibrado trivial.
\end{remark}

\begin{question}\leavevmode
Me queda claro que si \(f\) es constante, la fibra de \(f^*E\) siempre es \((f^*E)_p \cong E_{f(*)}\)…
\end{question}

\begin{remark}\leavevmode
Pega  \(\xi \in\Gamma(f^* E)\). Então temos para \(p \in M\) um elemento \(\xi(p)=(p, \bar{\xi}(p)) \). Então olha
\[\begin{tikzcd}
	\bar{\xi} :M \arrow[r]\arrow[dr,"f"]&  E \arrow[d,"\pi"]\\& N
\end{tikzcd}\]
então essas seções se chamam de \(\mathfrak{X}_f \cong \Gamma(f^*(E))\) \textit{\textbf{seções ao longo de \(f\)}}.
\end{remark}
Entonces el punto es que, por construcción cada sección del pullback me da un elemento en el otro vb y de ahi quiero que la proyección me devuelva \(f\).

Note que para um campo vetorial \(X \in \mathfrak{X}(M)\) temos um campo \(f_*X\) que \textit{não é} um campo vetorial em \(N\). É um campo vetorial com base \(M\) e espaço total  \(f^*TN\). Parecidamente, se \(Y \in \mathfrak{X}(N)\), obtemos um campo sobre \(M\) com valores em \(f^*TN\) mediante \(Y \circ f\).
\begin{defn}\leavevmode
Dos campos \(X \in \mathfrak{X}(M)\) e \(Y \in \mathfrak{X}(N)\) están \textit{\textbf{\(f\)-relacionados}} \(X \overset{f}{\sim}Y\) se \(Y \circ f =f_*X\) donde \(f:M \to N\). Pero pérame porque a mí me habían dicho que no siempre \(f_*X\) está bien definido. Ah, porque aquí \(f_*X\) es un campo \textit{ao longo de \(f\)}; así \textit{siempre} está bien definido. Entonces tiene sentido la definición y el ejercicio:
\end{defn}

\begin{exercise}\leavevmode
	Pegue \(X_i \in \mathfrak{X}(M)\), \(Y_i\in \mathfrak{X}(N)\), \(i=1,2\). Mostre que  \[X_i \overset{f}{\sim}Y_i\implies [X_1,X_2] \overset{f}{\sim}[Y_1,Y_2]\]
\textbf{Hint.} Pensa que um campo é uma coisa que pega uma função e me da uma função.
\end{exercise}

\begin{proof}[Solução]\leavevmode
	Queremos ver que
	\[f_*[X_1,X_2]=[Y_1,Y_2]\circ f \in \mathfrak{X}_f\]
i.e. que esses campos são iguais \textit{como campos vetoriais ao longo de \(f\)}, que é um negócio bem estranho porque, de novo, o espaço base é \(M\) e o espaço total é \(f^* TN\) (que é bem parecido a \(TN\) mas  não é \(TN\)---pode ser incluído eu acho).

E isso é super importante porque esclarece o jeito de proceder que é: pega \(p \in M\) e \(g \in \mathcal{F}(N)\). Beleza então temos
	\begin{align*}
	\Big([Y_1,Y_2]\circ f\Big)_p(g)&=Y_{1,f(p)}\Big(Y_2(g)\Big)-Y_{2,f(p)}\Big(Y_1(g)\Big)\qquad \text{blz} \\
				       &\overset{\operatorname{hip}}{=}f_*X_{1,p}\Big(Y_2(g)\Big)-f_*X_{2,p}\Big(Y_1(g)\Big)\\
&\overset{\operatorname{hip}}{=}f_*X_{1,p}\Big(f_*X_2(g)\Big)-f_*X_{2,p}\Big(f_*X_2(g)\Big)\\
&=f_*[X_1,X_2]_p(g).
	\end{align*}
\end{proof}

\subsection{Grupos de Lie}

\begin{defn}\leavevmode
Um \textit{\textbf{grupo de Lie}} é um grupo \(G\) que é uma variedade diferenciável tal que
\[\cdot :G \times G \to G \qquad \qquad \leavevmode^{-1}:G \to G\]
são diferenciaveis.
\end{defn}

Os grupos de Lie tem um monte de difeomorfismos dados pela multiplicacão a esquerda: \(h \in G \rightsquigarrow L_h:G \to G\), \(L_h(g)=h\cdot g\). Como \(L_{h^{-1}}\circ L_h=\operatorname{Id}\), \(L_h \in \operatorname{Dif}G\).

\begin{exercise}\leavevmode
\(v \in T_eG\), \(X_v(g)=d(L_g)_e(v) \in T_gG\), \(\implies\) \(X_v \in \mathfrak{X}(G)\). \textbf{Note} que vai precisar usar que o produto do grupo é diferenciável.
\end{exercise}

\begin{proof}[Solução]\leavevmode
Basta mostrar que, pegando qualquer vizinhança coordenada de qualquer ponto \(g \in G\), as funções coordenadas de \(X_v\) são diferenciáveis.

Pegue um sistema de coordenadas em  \(g \in G\), digamos \((U,x)\). Como  \(L_g\) é um difeomorfismo, obtemos um sistema de coordenadas \((L_{g^{-1}}(U),x')\) de \(e \in G\). Suponha que \(v=\sum v^i\partial_i\) nessas coordenadas. Então
\begin{align*}
	(d_eL_g)v&=(d_eL_g)\left(\sum v^i\partial_i\right) \\
	&=\sum (v^i \circ L_g)d_eL_g\partial_i
\end{align*}
Então essas funções coordenadas são suaves: para \(h \in G\) temos
\[(v^i \circ L_g)(h)=v^i(gh)\]
que é suave porque é a composição de duas funções suaves, e porque o produto do grupo de Lie é suave.
\end{proof}
E aí fica que uma base \(\{v_i\}\subset T_eG\) nos da uma base global de seções. Em outras palavras, o fibrado tangente de um grupo de Lie é trivial. Isso é rarísimo, uma variedade com fibrado tangente trivial, se chama variedade paralelizável.

\begin{remark}\leavevmode
\(\forall g \in G, X_v \overset{L_g}{\sim}X_v\) para todo \(v \in T_eG\). Acho que é por regra da cadeia. Queremos ver
que em todo ponto \(g \in G\), \[\Big((L_g)_*(X_v)\Big)_h=(X_v)_h\]
então fica que
\begin{align*}
\Big((L_g)_*(X_v)\Big)_h&=\Big((d_{g^{-1}h}L_g)(d_eL_{g^{-1}h})v\Big)_h=\Big(d_e(L_g\circ L_{g^{-1}h})v\Big)_h=\Big(d_eL_{h}v\Big)_h=(X_v)_h
\end{align*}
Mas ainda, se um campo vetorial \(X\) está \(L_g\) relacionado com ele mesmo para todo \(g \in G\) (isso se chama ser \textit{\textbf{invariante à esquerda}}), então ele é um \(X_v\) para algum \(v\). Conta:
\[v:=X_e \implies X_h=(L_{h,*}X)_h=(d_eL_{h}X_e)_h=d_eL_hv.\]
Então ai fica essa equivalência, e ademais, se pegamos \(v,w \in T_eG\) podemos pensar em \(X_v,X_w\), e \textit{definimos} \(X_{[v,w]}:=[X_v,X_w]\). E ai obtemos a \textit{\textbf{álgebra de Lie}} de \(G\), que é \((T_eG,[,]):=\mathfrak{g}\).
\end{remark}

\begin{thing6}{Mais um}\leavevmode
Pegue \(X \in \mathfrak{g}\) e \(\gamma\) curva integral de \(X\) passando por \(e\), i.e. \(\gamma(0)=e\). Prove que
\begin{enumerate}
\item Se \(\varphi_t\) é o fluxo de \(X\) \(\implies\) \(L_g \circ\varphi_t = \varphi_t\circ L_g\), \(\varphi_t = R_{\gamma(t)}\).
\item \(\gamma\) é homomorfismo de grupos \(\mathbb{R} \to G\). Isso permite definir \(\operatorname{exp}^G: \mathfrak{g} \to G\) dada por \(\operatorname{exp}^G(X)=\gamma(1)\). Prove que \(\operatorname{exp}^G(tX)=\gamma(t)\).

\textbf{Hint.} O último implica os outros.
\end{enumerate}
\end{thing6}

\begin{proof}[Solução]\leavevmode
\begin{enumerate}
\item Pegue \(h \in G\). O único que sei de \(\varphi_t\) é que
	\[\frac{d}{dt}\Big|_{t=0}\varphi_t(h)=X_h\]
E quero ver que
\[\varphi_t(gh)=(\varphi_t \circ L_g)(h)\overset{\text{quero} }{=}(L_g \circ \varphi_t)(h)=g\varphi_t(h)=L_g(\varphi_t(h))\]
Então derivo:
\begin{align*}
\frac{d}{dt}\Big|_{t=0}\varphi_t(gh)&=X_{gh}\overset{\operatorname{def}}{=}d_eL_{gh}(X_e)\overset{\substack{\text{chain}  \\ \text{rule}} }{=}d_hL_gd_eL_h(X_e)\overset{\operatorname{def}}{=}d_hL_gX_h=d_hL_g\left(\frac{d}{dt}\Big|_{t=0}\varphi_t(h)\right) 
\end{align*}
de forma que as derivadas das coisas que quero que sejam iguais coincidem. Avaliando em \(t=0\) vemos que as funções devem ser iguais.

A comprovação de que \(\varphi_t=R_{\gamma(t)}\) é análoga: definindo \(X:=X_{X}\) (e é assim porque \(X\in \mathfrak{g}\)), tenho duas funções
\begin{align*}
\begin{aligned}
R_{\gamma(t)}: G &\longrightarrow G \\
	g &\longmapsto g\cdot\gamma(t)
\end{aligned}\qquad \begin{aligned}
\varphi_t: G &\longrightarrow G \\
g &\longmapsto \substack{\text{integro \(X_g\)}\\\text{e avanço \(t\)} }
\end{aligned}
\end{align*}
Derivo:
\begin{align*}
	\frac{d}{dt}\Big|_{t=0}g\cdot \gamma(t)&=\frac{d}{dt}\Big|_{t=0}(L_g \circ \gamma)(t)\overset{\substack{\text{chain}\\\text{rule}}}{=}d_eL_g\cdot \gamma'(0)=X_g=\frac{d}{dt}\Big|_{t=0}\varphi_t(g)
\end{align*}
avaliando em \(t=0\) obtemos a igualdade.

\item  Talvez tô errado mas acho que é o mesmo: queremos ver que
	\[\gamma(t_1+t_2)\overset{\text{quero}}{=}\gamma(t_1)\gamma(t_2)\overset{\operatorname{def}}{=}L_{\gamma(t_1)}\gamma(t_2)\]
então derivo respeito a \(t_2\) 
\begin{align*}
\frac{d}{dt_2}\Big|_{t_2=0}L_{\gamma(t_1)}\gamma(t_2)&\overset{\substack{\text{chain}\\\text{rule}}}{=}d_{\gamma(0)}L_{\gamma(t_1)}\gamma'(0)=d_eL_{\gamma(t_1)}X_e\\&=X_{\gamma(t_1)}\overset{\substack{\text{\(\gamma\) curva}  \\ \text{integral} }}{=} \gamma'(t_1)=\frac{d}{dt_2}\Big|_{t_2=0}\gamma(t_1+t_2)
\end{align*}
e de novo, avaliando em \(t_2=0\) obtemos a igualdade.

Por fim, para o último exercício queremos achar uma curva integral de \(tX\), \(t\) fixo, i.e.
\[\tilde{\gamma}:\mathbb{R}\to G\qquad \text{tal que} \qquad \tilde{\gamma}'(s)=(tX)_{\tilde{\gamma(s)}}\forall s \in \mathbb{R}.\]
Sinto no cora que 
\[\tilde{\gamma}(s):=\gamma(ts)\qquad \text{vai dar certo.} \]
Então derivo
\[\frac{d}{ds}\Big|_{s=s}\tilde{\gamma}(s)=\frac{d}{ds}\Big|_{s=s}\gamma(ts)=\gamma'(ts)t=X_{\gamma(t s)}t=(tX)_{\gamma( ts)}=(tX)_{\tilde{\gamma}(s)}\]
olha só
\[\operatorname{exp}^G(tX)=\tilde{\gamma}(1)=\gamma(t).\]

\end{enumerate}
\end{proof}

\section{Aula 3: A primeira aula}

\subsection{Lembrando geometria diferencial de curvas e superfícies}

A história começa com o Gauss em \(1827\).

A geometria de superfícies se faz assim. Pega \(p \in M^2 \subset \mathbb{R}^3\). Pode botar uma métrica canônica usando a inclusão \(i\), i.e.
\begin{align*}
	\left<\cdot,\cdot\right>_p: T_pM^2 \times T_pM^2 &\longrightarrow \mathbb{R} \\
	(v,w) &\longmapsto \left<i_{*,p}v,i_{*,p}w\right>_p
\end{align*}
Também pode só derivar curvas na superfície, obtendo vetores em \(\mathbb{R}^3\), e usando o produto usual de \(\mathbb{R}^3\).

O Gauss definiu o mapa normal \(N(p)\), derivando ele para obter
\[A:=d_p N : T_pM ^2 \to T_pM^2\]
que ressoltou ser um endomorfismo autoadjunto (respeito a aquela métrica que a gente falou). Dai apareceram
\[\operatorname{tr}A=H\qquad \text{curvatura média} \]
\[\det A=K\qquad \text{curvatura Gaussiana} \]
E ai o Gauss descobriu que \(K\) depende s ó da métrica, i.e. \(K=K(\left<\cdot,\cdot\right>)\). A curvatura média não. (E.g. um plano pode ser mergulhado em \(\mathbb{R}^3\) como um cilindro, \(K\) fica igual, enquanto \(H\) muda.)

\subsection{Riemann}

\begin{defn}\leavevmode
Uma \textit{\textbf{variedade Riemanniana}} é uma variedade diferenciável \(M^n\) junto com um tensor
\[\left<\cdot,\cdot\right>: \mathfrak{X}(M) \times \mathfrak{X}(M) \longrightarrow \mathcal{F}(M)\]
simétrico e positivo definido. De acordo com aquele exercício, isso significa que para cada \(p \in M\) temos uma forma bilinear simétrica positiva definida \(\left<\cdot,\cdot\right>_p:T_pM \times T_pM \longrightarrow \mathbb{R}\).

A variedade é \textit{\textbf{semi-Riemanniana}} se, em lugar de positivo definido, o tensor é não degenerado, i.e. \(\forall v \in T_pM\), se \(\left<v,w\right>_p=0\) \(\forall w \in T_pM\), então \(v =0\). Nesse caso, definimos o \textit{\textbf{índice}} da forma como sendo
\[i(\left<\cdot,\cdot\right>_p:=\operatorname{max}\left\{ \dim \mathbb{L} \overset{\text{sub} }{\subset} T_pM:\left<\cdot,\cdot\right>|_{\mathbb{L}\times\mathbb{L}}<0\right\} \]

Bom pegue um sistema coordenado \((x,U)\). Podemos definir para \(Q \in M\)
\[g_{ij}(Q):=\left<\partial_i(Q),\partial_j(Q)\right> \in \mathcal{F}(U)\]
i.e.
\[(g_{ij})_Q:U \longrightarrow \mathsf{GL}(n,\mathbb{R}) \cap \operatorname{Sym}(n)\]
ou seja, a variedade é Riemanniana quando essas funções são positivas.

Se a variedade é Riemanniana temos uma norma \(\|v\|:=\sqrt{v,v} \). (Se não não.)
\end{defn}

\begin{remark}\leavevmode
A definição de variedade Riemanniana foi dada por Weil nos anos 30.
\end{remark}

\begin{defn}[Isometrias]\leavevmode
\(f:M \to N\). Primeiro note que podemos definir o pullback de qualquer tensor. Para \(f:M \to N\) e \(T\) tensor da forma \(T: \mathfrak{X}(N)\times \mathfrak{X}(N) \longrightarrow \mathcal{F}(N)\), definimos
\[f^*(T)_p(u,v)_p:=T(f(p))(f_*u,f_*v)\]
Note que de graça é simétrico se o tensor em \(N\) é simétrico.

Para ver positivo definido temos que o pullback é positivo definido \(\iff\) \(f\) é um mergulho. Prova: considera a norma. A norma de \(f_{*,p}u\) é positiva \(\iff\) \(u \neq 0\). Para asegurar que a preimagem desse vetor também  não é zero precisamos que seja mergulho.

\(f:(N^n,\left<\cdot,\cdot\right>_N)\longrightarrow (M^m,\left<\cdot,\cdot\right>_M\) é uma \textit{\textbf{imersão isométrica}} se \(\left<\cdot,\cdot\right>_N = f^*\left<\cdot,\cdot\right>_M\).

Uma \textit{\textbf{isometria}} entre variedades Riemannianas é \(f:M^n \to \tilde{M}^n\) difeomorfismo e isometria (como imersão).

Uma \textit{\textbf{isometria local}} é um difeo local e isometria.
\end{defn}

\begin{remark}[Isomorfismos canônicos]\leavevmode
Para qualquer espaço vetorial \(\mathbb{V}\), temos um \textit{isomorfismo canônico} (i.e. não depende de escolha de base) \(\mathbb{V}^n \cong T_p\mathbb{V}^n\) dado por
\[\mathbb{V}^n \ni v \longmapsto \alpha_{p,v}'(0),\qquad  \alpha_{p,v}(t)=p+tv\]

Tem outro isomorfismo canônico: \(M\ni p,M' \ni p'\),
\[T_{(p,p')}(M \times M') \cong T_pM \times T_{p'}M'\]
\[w \longmapsto(\pi_{*,(p,p')}(w),\pi'_{*,(p,p')}(w)\]
onde
\[\begin{tikzcd}
&M\times M'\arrow[dl,"\pi",swap]\arrow[dr,"\pi'"]\\
M&&M'
\end{tikzcd}\]
\begin{exercise}\leavevmode
Mostre que o inverso desse mapa ai é
\[(\pi_{*,(p,p')}(w),\pi'_{*,(p,p')}(w) \longmapsto(i_p)_{* p'}(v') + (i'_{p'})_{* p}(v)\]
com as inclusoes naturais.
\end{exercise}
\end{remark}

\begin{thing7}{Cuidado}[Acho]\leavevmode
Nem sempre é certo que \(T(M \times M') \cong \)``\(TM \times TM\)". Porque as funções coordenadas dependem de dois parámetros: \(Z \in \mathfrak{X}(M \times M')\), \(Z=X+X'\),
 \[\sum a_i(p,p')\partial_i|_{p}+\sum_{j}b_j(p,p') \partial_j\]
\end{thing7}

\begin{example}\leavevmode
\begin{enumerate}
\item \(\mathbb{R}^n\) com o produto canônico usando o isomorfismo canónico de \(\mathbb{R}^n \cong T_p \mathbb{R}^n\) acima.

\item (Grupo de Lie.) \(h \in G\), \(L_h\) traslação a esquerda. Usemos as traslações a esquema para definir uma métrica em \(G\). Pegue qualquer produto interno \(\left<\cdot,\cdot\right>_e\) em \(\mathfrak{g}\). E traslade:
	\begin{align*}
	\left<\cdot,\cdot\right>_h&:=L_h^*\left<\cdot,\cdot\right>_e
	\end{align*}
	i.e.,
	\[\left<v,w\right>_h=\left<dL_{h^{-1}}(v),dL_{h^{-1}}(w)\right>_e\]
\begin{exercise}\leavevmode
\begin{enumerate}
\item Isto define uma métrica Riemanniana em \(G\).
 \item \(\forall  X,Y \in \mathfrak{g}\), \(\left<X,Y\right>=\operatorname{cte}\).
\item \(\left<\cdot,\cdot\right>\) é invariante a esquerda, i.e. \(\forall  h \in G\), \(L_h\) é isometria de \((G,\left<\cdot,\cdot\right>)\).
\begin{remark}\leavevmode
Essa métrica é invariante a \textit{a esquerda}. Nem tem que ser invariante a direita.
\end{remark}
\begin{remark}\leavevmode
Vai ter um exercício de do Carmo dizendo que se \(G\) é compacto vai ter uma métrica bi-invariante, i.e. o promédio.
\end{remark}
\end{enumerate}
	{\color{7}dani:} parece que sempre que temos uma ação homogênea podemos transportar a métrica de \(\mathfrak{g}\) pra todos lados.
\end{exercise}	
\item \(M^n \subset \mathbb{R}^{n+[}\) subvariedade regular (=inclusão é um mergulho). Podemos fazer o que Gauss fez:
	\[\left<u,v\right>_p:=\left<i_{*,p}u,i_{*,p}v\right>^{\mathbb{R}^{n+p}}_{\operatorname{can}}\]
\end{enumerate}
\end{example}

\begin{question}\leavevmode
Será que toda variedade Riemanniana admite um mergulho isométrico em algum \(\mathbb{R}^{n+p}\)? Quem é \(p\)?
\end{question}

\begin{thing7}{Nash}\leavevmode
O caso \(C^1\) é fácil, 
\end{thing7}

\begin{question}[dani]\leavevmode
Em topo dif vimos primeiro uma prova de que pode mergulhar qualquer variedade em um \(\mathbb{R}^N\) com \(N\) muito grande. Depois os teoremas de Whitney mostrarem que \(N\) pode ser mais o menos pequeno. Aqui podemos mostrar que o mergulho/imersão existe para \(N\gg\) mais o menos facilmente?
\end{question}

\begin{prop}[Existência de métricas Riemannianas]\leavevmode
Se \(M\) é uma variedade diferenciável, existe uma métrica Riemanniana em \(M\).
\end{prop}

\begin{thing6}{What}\leavevmode
que em toda variedade tem um aberto denso difeomorfo a uma bola.
\end{thing6}

\begin{proof}\leavevmode
Pegue um atlas \(\{(X_\lambda,U_\lambda)\}\) localmente finito para usar uma partição da unidade subordinada  \(\{\rho_\lambda\}\). Pega qualquer carta e puxe a métrica de \(\mathbb{R}^n\), i.e. se \(x_\lambda:U_\lambda \longrightarrow \mathbb{R}^n\), \(x^* _\lambda \left<\cdot,\cdot\right>_{\operatorname{can}}\) é uma métrica riemanniana em \(U_\lambda\).

\(\rho_\lambda x^* _\lambda \left<\cdot,\cdot\right>_{\operatorname{can}}\). Fica um tensor simétrico \textit{semi} positivo definido, i.e. \(\geq 0\).

No final define para \(p \in M\), \(v \in T_pM\),
\[\left<v,v\right>:=\sum_{\lambda | \rho_\lambda(?)>0}\rho_\lambda(p)\|x_\lambda)_{*,p}v\|^2>0\]
i.e. fica positiva.
\end{proof}

Definimos o angulo entre \(v,w \in T_pM\) como satisfazendo
\[\cos (\operatorname{a n g u l o}(v,w)):=\frac{\left<v,w\right>}{\|v\|\|w\|}\]

\begin{thing7}{Soft exercise}\leavevmode
Ortogonalize Gram-Schmidt uma base \(\{v_i\}\) de um espaço vetorial \(\mathbb{V}\) para obter uma base ortonormal \(\{e_i\}\) (com a mesma orientação).
\end{thing7}

\begin{remark}\leavevmode
	O processo pode ser feito igualzinho para campos vetoriais: se \(X_1,\ldots,X_n\) é uma base local de campos, \(\exists !\) base ortonormal de campos \(\{e_i\}\). \textbf{Cuidado:} em geral, o colchete desses campos não é zero, i.e.  \([e_i,e_j] \neq 0\).
\end{remark}

\begin{prop}[Elemento de volume]\leavevmode
\(M^n\) variedade Riemanniana orientada  \(\implies\) \(\exists !\) \(\omega \in \Omega^{n}(M^n)\) tal que
\[\omega(\text{bon+} )=1\]
bon+=base ortonormal orientada.
\end{prop}

\begin{thing6}{Lembre}\leavevmode
Para duas top-forms, uma se expressa como a outra multiplicando pelo determinante da mudança de base.
\end{thing6}

\begin{proof}\leavevmode
Como \(M\) é orientada, sabemos que \(\exists  \sigma \in \Omega^{n}(M^n)\) positiva. Buscamos a função \(f\) tal que \(\omega=f\sigma\). Pega um ponto, bases coordenadas \(\{\partial_i\}\) e ortonormaliza para obter \(\{e_i\}\). Como queremos que
\[\omega(e_1,\ldots,e_n)\overset{\text{quero}}{=}1\overset{\text{quero}}{=}f|_{U}\sigma(e_1,\ldots,e_n)\]
só tem um jeito de definir \(f\):
\[f|_{U}=\sigma(e_1,\ldots,e_n).\]
E isso determina por completo \(f\) como uma função global suave, e portanto temos \(\omega\).
\end{proof}


\end{document}
