\input{/Users/daniel/github/config/preamble-por.sty}%available at github.com/danimalabares/config
%\input{/Users/daniel/github/config/thms-por.sty}%available at github.com/danimalabares/config

\newcommand{\rightlooparrow}{\mathbin{
    \vbox{\openup-10.25pt\halign{\hss$##$\hss\cr\circ\cr\longrightarrow\cr}}
}}

\begin{document}
\bibliographystyle{alpha}

\begin{minipage}{\textwidth}
	\begin{minipage}{1\textwidth}
		\hfill Daniel González Casanova Azuela
		
		{\small Prof. Luis Florit\hfill\href{https://github.com/danimalabares/rg}{github.com/danimalabares/rg}}
	\end{minipage}
\end{minipage}\vspace{.2cm}\hrule

\vspace{10pt}
{\huge Exercícios de Geometria Riemanniana}
\tableofcontents

\section{Lista 1}

\subsection{Revisão}

\begin{thing4}{Exercício 1}\label{exer:1}\leavevmode
Dada uma subvariedade \(M \subseteq \tilde{M}\) uma subvariedade mergulhada e \(X \in \mathfrak{X}(M)\). Mostre que existe um aberto \(U \subset \tilde{M}\) contendo \(M\) e um campo \(\tilde{X} \in \mathfrak{X}(U)\) tal que \(\tilde{X}|_{M}=X\). Caso \(M\) seja subconjunto fechado de \(\tilde{M}\), prove que \(U\) pode ser tomado igual a \(\tilde{M}\). Se \(M\) não é subconjunto fechado de \(\tilde{M}\), pode não existir extensão de \(X\) definida em todo \(\tilde{M}\).
\end{thing4}

\begin{proof}[Solução]\leavevmode
Acho que a prova canônica é tomar coordenadas de subvariedade de \(M \subset \tilde{M}\), i.e. onde \(M\) está dada localmente como o lugar onde se anulam as últimas \(n-m\) funções coordenadas. 

Pegamos uma vizinhança rectificante \(U\) de \(X\) em \(p \in M\), i.e. \(X=\partial_1\) em \(U\). Daí pega para cada vetor normal a exponencial, que percorre pela geodésica um pouqinho. Isso da uma vizinhança em \(\tilde{M}\)…
\end{proof}

\begin{thing4}{Exercício 2}\label{exer:2}\leavevmode
	Seja \(f:M^n \to N^m\) um mapa suave. Os campos \(X \in \mathfrak{X}(M)\) e \(\tilde{X} \in \mathfrak{X}(N)\) são ditos \(f\)-relacionados se \(df_pX_p=\tilde{X}_{f(p)}\), \(\forall  p \in M\). Mostre que se os campos \(X,Y \in \mathfrak{X}(M)\) são, respetivamente, \(f\)-relacionados com \(\tilde{X},\tilde{Y} \in \mathfrak{X}(N)\) então \([X,Y]\) é \(f\)-relacionado com \([\tilde{X},\tilde{Y}]\).
\end{thing4}

\begin{proof}[Solução]\leavevmode
	\textbf{Intento 2.} \(s_1 \in \Gamma(\tau_N)\) está \textit{\textbf{\(f\)-relacionado}} com \(s \in \Gamma(\tau_M)\) se \(s=s_1\oplus s^\perp\) para algum \(s^\perp\in\nu\). Queremos ver que se \(s\overset{f}{\sim}s_1\) e \(t \overset{f}{\sim}t_1\), \([s,t]\overset{f}{\sim}[s_1,t_1]\), ou seja \([s,t]=[s_1,t_1]\oplus [s,t]^\perp\) onde \([s,t]^\perp\) é um vetor em \(\nu\) cuja cara não  é muito importante.
	\begin{align*}
		[s,t]&=[s_1\oplus s^\perp,t_1\oplus t^\perp]=[s_1,t_1]+\underbrace{[s_1,t^\perp]}_{=0}+\underbrace{[s^\perp,t_1]}_{=0}+\underbrace{[s^\perp,t^\perp]}_{\in \nu}
	\end{align*}
{\color{4}Falta un argumentín para ver que esos colchetes se anulan…}

\textbf{Intento 1 (incompleto).} Pegue \(p \in M\). Queremos ver que
\[(f_{*}[X,Y])_p\overset{\text{quero}}{=}[\tilde{X},\tilde{Y}]_{f(p)}.\]
Pegue \(g \in \mathcal{F}(N)\).
\begin{align*}
	[\tilde{X},\tilde{Y}]_{f(p)}&\overset{\operatorname{def}}{=}\tilde{X}_{f(p)}(\tilde{Y}g)-\tilde{Y}_{f(p)}(\tilde{X}g)\\
	&\overset{\operatorname{hip}}{=}f_{*,p}(X_p)(\tilde{Y}g)-f_{*,p}(Y_p)(\tilde{X}g)\\
	&=X_{p}\Big((\tilde{Y}g)\circ f\Big)-Y_p\Big((\tilde{X}g)\circ f\Big)\\
	&\overset{\operatorname{hip}}{=}X_p\Big(\big(f_{*,p}(Y)\big)g\circ f)\Big)-Y_p\Big(\big(f_{*,p}(X_p)\big)g \circ f\Big)\\
\end{align*}
\end{proof}

\begin{thing4}{Exercício 3}\label{exer:3}\leavevmode
Seja \(\pi:M \to N\) uma submersão sobrejetiva. Dado \(Y \in \mathfrak{X}(N)\), mostre que existe \(X \in \mathfrak{X}(M)\) tal que \(X\) é \(\pi\)-relacionado com \(Y\).
\end{thing4}
\begin{proof}[Solução]\leavevmode
O resultado segue de que \(\tau_M\cong\pi^*\tau_N\oplus \nu\), tomando \(X:=Y\oplus 0\).
\end{proof}

\begin{thing4}{Exercício 4}[Fibrado pullback]\label{exer:4}\leavevmode
Suponha que \(M^n\), \(N^m\) são variedades suaves, \(\pi:E \to M\) é um fibrado vetorial suave de posto \(k\) e \(f:N \to M\) é um mapa suave. Considere o espaço
\[f^*E=\{(p,e) \in N \times E:f(p)=\pi(e)\},\]
e \(\tilde{\pi}:E \to N\) a projeção na primeira coordenada. Mostre que \(f^*E\) tem uma estrutura de variedade suave de forma que a tripla \(\tilde{\pi}:f^*E\to N\) é um fibrado vetorial suave de posto \(k\).
\end{thing4}

\begin{proof}[Solução]\leavevmode
Para mostrar que \(\tilde{\pi}\) é um fibrado vetorial devemos dar trivializações locais. Pegue um ponto \(p \in M\) e uma vizinhança trivializante de \(E\) perto de \(f(p)\), i.e. um aberto \(U \ni f(p)\) e um difeomorfismo \(h:\pi^{-1}(U)\xrightarrow{\cong}U\times \mathbb{R}^k\). Pegue também um aberto \(V \ni p\) tal que \(f(V) \subset U\). Defina
\begin{align*}
	h_1: \tilde{\pi}^{-1}(V) &\longrightarrow V \times \mathbb{R}^k \\
	(q,v) &\longmapsto (q,\pi_2\circ h(f(q),v))
\end{align*}
Como estamos usando a estrutura de fibrado vetorial de \(E\), segue imediatamente a coleção de funções desse tipo formam um atlas trivializante de \(f^*E\).
\end{proof}

\subsection{Métricas Riemannianas}

\begin{thing4}{Exercício 6}\label{exer:6}\leavevmode
Seja \((N^n,g)\) uma variedade Riemanniana e \(M^m \subset N\) uma subvariedade mergulhada. Mostre que para todo \(p \in M\) existe uma vizinhança aberta \(U \subset N\) de \(p\) e campos vetoriais \(E_1,\ldots,E_n\) em \(U\) tal que \(E_1(q),\ldots,E_n(q)\) é uma base ortonormal de \(T_q N\) para todo \(q \in U\) e \(E_1(r),\ldots,E_m(r)\) são tangentes a \(M\) para todo \(r \in U \cap M\).
\end{thing4}

\begin{proof}[Solução]\leavevmode
\textbf{(Intento 1.)}Pegue \(p \in M\) e uma vizinhança aberta de \(U \subset N\) de \(p\) tal que \(U \cap M\) é suficientemente pequeno como para ter um marco ortonormal \(\{E_i\}_{i=1}^n\). Considere esses campos como campos tangentes a \(N\). Usando o exercício 1 podemos estender esses campos a uma vizinhança de \(U \subset N\). Aplicando Gram-Schmidt obtemos um marco ortonormal de  \(\mathfrak{X}(U)\).

\textbf{(Intento 2, \cite{mc} thm. 3.3, p. 36.)} Take orthonormal frames  \(\{E_i\}_{i=1}^m\subset\mathfrak{X}(U \cap M)\) and \(\{E_i'\}_{i=1}^n \subset \mathfrak{X}(U)\). Notice that the matrix \((E_i\cdot E_j')\) has rank \(m\) at \(p\). (I think that two orthonormal frames are related up to an orthogonal matrix.) Suppose that the first \(m\) columns are linearly independent at \(p\). Then there is an open neighbourhood \(V\) of \(p\) where the first \(m\) columns of this matrix are linearly independent. Then a slightly confusing part arguing that \(E_1,\ldots,E_m,E_{m+1}',\ldots,E_{n}'\) are linearly independent in \(V\). Then apply Gram-Schmidt. And that's it.

Then Milnor shows that this is a vector bundle called the \textit{\textbf{orthogonal bundle}}. The lance is that the orthonormal frame we have found gives the local trivialization. For a subbundle \(\xi \subset \eta\) define the fiber of the orthogonal complement of \(\xi\) by \(F_b(\xi^\perp):=F_b(\xi)^\perp\) with respect to the metric of \(\eta\). Define local trivializations by
\begin{align*}
	\overline{h}: \overline{\pi}^{-1}(U) &\longrightarrow U\times \mathbb{R}^{n-m} \\
	 \Big(q,\sum x_iE_i\Big)&\longmapsto (q,x_{m+1},\ldots,x_m)
\end{align*}

\end{proof}

\begin{thing4}{Definição 1}\label{def:1}\leavevmode
Sejam \((M^m,g_M)\) e \((N^n,g_N)\) variedades Riemannianas. Seja \(F:M \to N\) uma submersão. Dizemos que \(F\) é uma \textit{\textbf{submersão Riemanniana}} quando para todo \(p \in M\), \(DF: \ker (DF)^\perp\to T_{F(p)}N\) é uma isometría linear. Em outras palavras, sempre que \(v, w \in T_pM\) são perpendiculares ao núcleo de \(DF\), vale
\[g_M(v,w)=g_N(DF(v),DF(w)).\]
\end{thing4}

\begin{thing4}{Exercício 7}\label{exer:7}\leavevmode
Seja \((M^n,g)\) uma variedade Riemanniana. Suponha que existe um grupo de Lie \(G\) agindo por isometrias em \((M,g)\) de tal forma que \(M/G\) admite uma estrutura de variedade suave, onde a projeção \(\pi:M \to M/G\) é uma submersão. Mostre que existe uma métrica Riemanniana \(\overline{g}\) em \(M/G\) tal que \(\pi:(M,g) \to (M/G,\overline{g})\) é uma submersão Riemanniana.
\end{thing4}
\begin{proof}[Solução]\leavevmode
	\textbf{(Seguindo notação e ideias de \cite{mc}.)} Fazemos assim para definir a métrica em \(G/M\). Primeiro lembre que \(\tau_{G/M} \cong \pi^*\tau_{M/G}\). Considere o fibrado \(\nu\) normal a \(\pi^*\tau_{M/G}\), que é um fibrado sobre \(M\) satisfazendo \(\pi^*\tau_{G/M}\oplus \nu\cong\tau_M\). Então qualquer vetor tangente a \(M/G\) pode ser pensado como um vetor tangente a \(M\) se anulamos a parte normal dele, mostrando que podemos usar a mesma métrica em \(M\) para introduzir uma métrica em \(G/M\).

	Para resolver o exercício devemos analisar como age \(\pi_*\) em \(\tau_M\) quando este es visto como soma direita \(\pi^*\oplus \nu\): \(\pi_*(v_1 \oplus v^\perp)=v_1\). Daí segue trivialmente que \(\ker \pi:=\kappa\subset\nu\). Conversamente se \(v_1 \oplus v^\perp \in \kappa\), fazemos para \(w \in \pi^*\)
	\begin{align*}
	(v_1 \oplus  v^\perp)\cdot w&=v_1 \cdot w+\cancelto{0}{v^\perp\cdot w}=\pi_*v_1\cdot\pi_*w=0.
\end{align*}
Então \(\kappa=\nu\), então \(\kappa^\perp\cong\pi^*\cong\tau_{M/G}\) isometricamente.

	\textbf{Intento 1 (errado).} Defina a seguinte métrica em \(M/G\):
\[g_{M/G}:=g_M|_{\pi^*\tau_{M/G}}\]
i.e. a restrição da métrica em \(M\) ao fibrado pullback de \(\tau_{M/G}:=T(G/M)\), que sabemos que é isomorfo (como fibrado) a \(\tau_{M/G}\).

Para ver que \(\pi:M\to M/G\) é uma submersão Riemanniana devemos mostrar que o complemento ortogonal de \(\kappa_\pi:=\ker(\pi)\) é isomorfo (como fibrado Riemanniano, i.e. isométrico como fibrado) a \(\tau_{M/G}\).

Como \(M\) é Riemanniana, o fibrado pullback tem um complemento ortogonal \((\pi^*\tau_{M/G})^\perp := \nu\). Basta mostrar que \(\nu \cong\kappa\) isometricamente.



\end{proof}

\section{Exercícios de \texttt{aulas.pdf}}

\begin{exercise}\leavevmode
Show that for a bi-invariant metric on a Lie Group, it holds that \(\operatorname{exp}_e=\operatorname{exp}^G\).
\end{exercise}

\begin{proof}[Solution]\leavevmode
After delving into the abyss of definitions, I think it boils down to showing that \(\nabla_{X_v} X_v=0\), where \(v \in \mathfrak{g}\). So we have to use that the metric is bi-invariant. But it's not necessarily Levi-Civita connection…
\end{proof}


\section{Exercícios do do Carmo}

\subsection{Capítulo 0}
\begin{thing4}{Exercise 2}\label{exer:2}\leavevmode
Prove que o fibrado tangente de uma variedade diferenciável \(M\) é orientável (mesmo que \(M\) não seja).
\end{thing4}

\begin{proof}[Solution]\leavevmode
Es porque la diferencial de los cambios de coordenadas está dada por la identidad y una matriz lineal. Sí, porque por definición las trivializaciones locales de \(TM\) preservan la primera coordenada \textbf{y} son isomorfismos lineales en la parte del espacio vectorial. Entonces queda que 
\[d(\varphi_U \circ \varphi_V^{-1})=\left(\begin{array}{@{}c|c@{}}
\operatorname{Id}&0\\
\hline
0&\xi \in \mathsf{GL}(n)\end{array}\right)\]
pero no estoy seguro de por qué \(\xi\) preservaría orientación, i.e. que tenga determinante positivo… a menos de que… 
\end{proof}

\begin{thing4}{Exercise 5}[Mergulho de \(P^2(\mathbb{R})\) em \(\mathbb{R}^4\) ]\label{exer:5}\leavevmode
Seja \(F:\mathbb{R}^3 \to \mathbb{R}^4\) dada por
\[F(x,y,z)=(x^2-y^2,xy,xz,yz),\qquad (x,y,z) = p \in \mathbb{R}^3.\]
Seja \(S^2 \subset \mathbb{R}^3\) a esfera unitária com centro na origem \(0 \in \mathbb{R}^3\). Oberve que a restrição \( \varphi:= F|_{S^2}\) é tal que \(\varphi(p)=\varphi(-p)\), e considere a aplicação \(\tilde{\varphi}:\mathbb{R}P^2 \to \mathbb{R}^4\) dada por
\[\tilde{ \varphi}([p])=\varphi(p),\qquad [p]\text{=clase de equivalência de \(p=\{p,-p\}\)} \]

Prove que
\begin{enumerate}[label=(\alph*)]
\item \(\tilde{\varphi}\) é uma imersão.
\item \(\tilde{\varphi}\) é biunívoca; junto com (a) e a compacidade de  \(\mathbb{R}P^2\), isto implica que \(\tilde{ \varphi}\) é um mergulho.
\end{enumerate}
\end{thing4}

\begin{proof}[Solution]\leavevmode
\begin{enumerate}[label=(\alph*)]
\item Considere a carta \(\{z=1\}\). A representação coordenada de \(\tilde{\varphi}\) vira
	\[(x,y) \longmapsto (x^2-y^2,xy,x,y)\]
cuja derivada como mapa \(\mathbb{R}^2 \to \mathbb{R}^4\) é
\[\begin{pmatrix} 2x& -2y\\y&x\\1&0\\0&1 \end{pmatrix} \]
que é injetiva. Agora pegue a carta \(\{x=1\}\). Então a representão coordenada de \(\tilde{ \varphi}\) vira
\[(y,z) \longmapsto (1-y^2,y,z,yz)\]
e tem derivada
\[\begin{pmatrix} -2y&0\\1&0\\0&1\\z&y \end{pmatrix} \]
que também é injetiva. Seguramente algo análogo acontece na carta \(\{y=1\}\).

\item \(\tilde{\varphi}\) é injetiva. Pegue dois pontos \(p_1:=[x_1:y_1:z_1]\) e \(p_2:=[x_2:y_2:z_2]\) e suponha que \(\tilde{\varphi}(p_1)=\tilde{\varphi}(p_2)\). I.e.,
	\[x_1^2-y_1^2=x_2^2-y_2^2,\qquad x_1y_1=x_2y_2,\qquad x_1z_1=x_2z_2, \qquad y_1z_1=y_2z_2\]
	Suponha primeiro que \(z_1 \neq  0\). Segue que
	\[x_1=\frac{z_2}{z_1}x_2, \qquad y_1=\frac{z_2}{z_1}y_2\]
	logo
	\[x_2^2-y_2^2=x_1^2-y_1^2=\left(\frac{z_2}{z_1}\right)^2(x_2^2-y_2^2)\implies z_2=z_1\implies x_1=x_2,\qquad y_1=y_2\]
	

	Em fim, uma imersão injetiva com domínio compacto é um mergulho porque é fechada: pegue um fechado no domínio, vira compacto, imagem é compacta, que é fechado. Pronto.
.
\end{enumerate}
\end{proof}

\begin{thing4}{Exercício 8}\label{exer:8}\leavevmode
\(\varphi:M_1\to M_2\) difeo local. Se \(M_2\) é orientável, então \(M_1\) é orientável.
\end{thing4}
\begin{proof}[Solução]\leavevmode
Defina: uma base \(\beta \subset T_pM\) é orientada se \(\varphi_*\beta\) é orientada em \(T_{\varphi(p)}M\). Tá bem definida porque \(\varphi\) é um difeomorfismo em \(p\), i.e. \(\varphi_*\) é isomorfismo. Para mostrar que é contínua à la Lee, qualquer vizinhança de um ponto \(p \in M_1\), a correspondente carta coordenada em \(\varphi(p)\), um marco coordenado nela e puxe (pushforward baix \(\varphi^{-1}\)) de volta para \(U\). Difeomorfismo e muito bom: o pushforward the campos vetoriais está bem definido. E por construção está orientado.
\end{proof}
\subsection{Capítulo 1}

\begin{thing4}{Exercise 1}\label{exer:1}\leavevmode
Prove que a aplicação antípoda \(A:S^n \to S^n\) dada por \(A(p)=-p\) é uma isometria de \(S^n\). Use este fato para introduzir uma métrica Riemanniana no espaço projetivo real \(\mathbb{R}P^{n}\) tal que a projeção natural \(\pi: S^n \to \mathbb{R}P^{n}\) seja uma isometria local.
\end{thing4}
\begin{proof}[Solution]\leavevmode
	Lembre que a métrica de \(S^n\) é a induzida pela métrica euclidiana, onde pensamos que \(T_pS^n \hookrightarrow T_p\mathbb{R}^{n+1}\). É claro que \(A\) é uma isometría de \(\mathbb{R}^n\), pois ela é a sua derivada (pois ela é linear), de forma que \(\left<v,w\right>_p=\left<-v,-w\right>_{A(p)}=\left<v,w\right>_{-p}\).

	É um fato geral que se as transformações de coberta preservam a métrica, obtemos uma métrica no quociente de maneira natural, i.e. para dois vetores \(v,w\in T_p\mathbb{R}P^n\) definimos \(\left<v,w\right>_p^{\mathbb{R}P^n}:=\left<\tilde{v},\tilde{w}\right>_{\tilde{p} \in \pi^{-1}(p)}\).

Para ver que a projeção natural é uma isometria local basta ver que a diferencial de \(A\) é um isomorfismo em cada ponto. Mas como ela é \(-A\), isso é claro.
\end{proof}

\begin{thing4}{Exercício 7}\label{exer:7}\leavevmode
Seja \(G\) um grupo de Lie compacto e conexo (\(\dim(G)=n\)). O objetivo do exercício é provar que \(G\) possui uma métrica bi-invariante. Para isto, prove as seguintes etapas:
\begin{enumerate}[label=(\alph*)]
\item Seja \(\omega\) uma \(n\)-forma diferencial em \(G\) invariante à esquerda, isto é, \(L_x^*\omega=\omega\), para todo  \(x\in G\). Prove que \(\omega\) é invariante à direita.

	\textit{Sugestão}: Para cada \(a \in Ga\), \(R_a ^*\omega\) é invariante à esqueda. Decorre daí que \(R_a ^*\omega=f(a)\omega\). Verifique que \(f(ab)=f(a)f(b)\), isto é, \(f:G \to \mathbb{R}\setminus\{0\}\) é um homomorfismo (contínuo) de \(G\) no grupo multiplicativo dos números reais. Como \(f(G)\) é um subgrupo compacto compacto e conexo, conclui-se que \(f(G)=1\). Logo \(R_a ^*\omega=\omega\).
\item Mostre que existe uma \(n\)-forma diferencial invariante à esquerda \(\omega\) em \(G\).
\item Seja \(\left<\cdot,\cdot\right>\) uma métrica invariante à esquerda em \(G\). Seja \(\omega\) uma \(n\)-forma diferencial positiva invariante à esqueda em \(G\), é defina uma nova métrica Riemanniana \(\left<\left<\cdot,\cdot\right>\right>\) em \(G\) por
\begin{align*}\left<\left<u,v\right>\right>_p&=\int_G\left<(d R_x)_yu,(d R_x)_yv\right>_{yx}\omega,\\
	&x,y \in G,\qquad u,v \in T_yG
\end{align*}
Prove que \(\left<\left<\cdot,\cdot\right>\right>\) é bi-invariante.
\end{enumerate}
\end{thing4}
\begin{proof}[Solução]\leavevmode
\begin{enumerate}[label=(\alph*)]
\item 
\item 
\item Vou usar outra notação. Suponha que \(g\) é uma métrica invariante à esquerda em \(G\). Definimos
	\[\tilde{g}:=\int_{x \in G}(R_x^*g)\omega\]
	como operador \(\mathfrak{X}(G)\times \mathfrak{X}(G) \longrightarrow \mathcal{F}(G)\).

Agora vamos ver que \(\tilde{g}\) é invariante à esquerda, i.e. queremos ver que para todo \(a \in G\),
\[\tilde{g}\overset{\text{quero}}{=}L_a^*\tilde{g}\overset{\operatorname{def}}{=}L^*_a \int_G(R_x^*g)\omega.\]
Vamos ver que o pullback \(L^*_a\) pode ``entrar na integral" e trocar de lugar com \(R^*_x\), daí o resultado segue porque \(g\) é \(L_a\)-invariante. As contas acabam sendo que
\begin{align*}
L_a ^*\int_G (R^*_xg)\omega&=\int_GL_a ^*R_x^*g \omega=\int_G (L_a \circ R_x)^*g\omega=\int_G(R_x \circ L_a)^*g\omega\\
&=\int_G R_x ^*L_a ^*g\omega=\int_GR_x^*g \omega=\tilde{g}
\end{align*}
Para ver que \(\tilde{g}\) também  é invariante à direita fazemos:
\begin{align*}
\tilde{g}&\overset{\text{quero}}{=}R_a ^*\tilde{g}\overset{\operatorname{def}}{=}R_a ^*\int_G(R_x^*)g\omega=\int_G R^* _aR_x^* g\omega=\int_G R_{ax}^*g\omega=\int_GR_x^*g\omega=\tilde{g}
\end{align*}
porque estamos integrando em todo \(G\) e \(G \mathbb{y} G\) transitivamente.

Para todo aquele que tem dúvida, aqui estão as contas da invarianza à esquerda super explicitas:

Fixe \(y \in G\) e \(u,v \in T_yG\). Temos que
\begin{align*}
	(L_a ^*\tilde{g})(u,v)&=L^* _a\left(\int_g(R_x^*g)\omega\right)(u,v)\\
	&=\left(\int_G (R_x^*g)\omega\right)\Big((L_a)_{*,a^{-1}y}u,(L_a)_{*,a^{-1}y}v\Big)\\
	&=\int_G(R_x^* g)\Big((L_a)_{*,a^{-1}y}u,(L_a)_{*,a^{-1}y}v\Big)\omega\\
	&=\int_Gg\Big((R_x)_{*,a^{-1}yx^{-1}}(L_a)_{*,a^{-1}y}u,(R_x)_{*,a^{-1}yx^{-1}}(L_a)_{*,a^{-1}y}v\Big)\omega\\
	&=\int_Gg\Big((R_x \circ L_a)_{*,a^{-1}yx^{-1}}u,(R_x\circ L_a)_{*,a^{-1}yx^{-1}}v\Big)\omega\\
	\text{associatividade em \(G\)} \qquad &=\int_Gg\Big((L_a \circ R_x)_{*,a^{-1}y x^{-1}}u,(L_a \circ R_x)_{*,a^{-1}yx^{-1}}v\Big)\omega\\
	&=\int_Gg\Big((L_a)_{*,a^{-1}yx^{-1}}(R_x)_{*,yx^{-1}}u,(L_a)_{*,a^{-1}yx^{-1}}(R_x)_{*,yx^{-1}}v\Big)\omega\\
	&=\int_G\Big((L_a)^* g\Big)\Big((R_x)_{*,yx^{-1}}u,(R_x)_{*,yx^{-1}}v\Big)\omega\\
\text{\(g\) invariante à esquerda} \qquad 	&=\int_Gg\Big((R_x)_{*,yx^{-1}}u,(R_x)_{*,yx^{-1}}v\Big)\omega\\
	&\overset{\operatorname{def}}{=}\tilde{g}(u,v).
\end{align*}
onde \(R_x \circ L_a=L_a \circ R_x\) por associatividade de produto no grupo.

\end{enumerate}
\end{proof}













\clearpage\bibliography{bib.bib}\end{document}
